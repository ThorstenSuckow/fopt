\chapter{Parallelität, Nebenläufigkeit und Verteilung}


\textbf{Parallelität} bzw. \textbf{Nebenläufigkeit (concurrency)} bedeutet: Mehrere Vorgänge laufen auf dem Rechner gleichzeitig ab.\\

Es wird unterschieden zwischen \textbf{echt gleichzeitig} und \textbf{scheinbar} gleichzeitig:
\begin{itemize}
    \item \textbf{echt gleichzeitig}: Mehrere Prozessoren (\textbf{Parallelität})
    \item \textbf{scheinbar gleichzeitig}: Ein Prozessor kümmert sich um mehrere Vorgänge gleichzeitig und schaltet auf sie in hoher Frequenz um (\textbf{Pseudoparallelität, Nebenläufigkeit})
\end{itemize}

Das Lernmaterial macht bewusst keine Unterscheidung zwischen Nebenläufigkeit und Parallelität, beide Begriffe schließen die echte und die Pseudoparallelität ein.

\begin{itemize}[label=$\rightarrow$]
    \item gleichzeitige Vorgänge auf einem Rechner: \textbf{Parallelität}
    \item gleichzeitige Vorgänge auf mehreren Rechnern: \textbf{Verteilung}
    \item \textbf{eng gekoppeltes System}: mehrere gekoppelte Prozessoren, die sich den Speicher teilen
    \item \textbf{lose gekoppeltes System}: verteiltes System, gekoppelte Prozessoren, ohne gemeinsamen Speicher, nutzen ein Kommunikationssystem zum Nachrichtenaustausch
\end{itemize}\\

Parallelität wird über Hard- und Software realisiert; das Betriebssystem spielt eine entscheidende Rolle und kann durch \textbf{Virtualisierung} die Anzahl der Prozessoren \textit{virtuell vervielfachen}.

\section{Programme, Prozesse und Threads}

In Zusammenhang mit Parallelität, Nebenläufigkeit und Verteilung muss unterschieden werden zwischen

\begin{itemize}
    \item \textbf{Programm}
    \item  \textbf{Prozess}
    \item \textbf{Thread} (``Ausführungsfaden``)
\end{itemize}\\

\subsection*{Prozess}
Der \textbf{Prozess} stellt im wesentlichen den Adressraum für \textbf{Programmcode} und \textbf{Daten} dar; ein Programm hat mindestens einen Thread; dieser kann weitere Threads starten; jeder Thread kann auf dieselben Objekte innerhalb eines Prozesses zugreifen, aber nicht auf Objekte außerhalb des Prozesses $\rightarrow$ \textbf{Isolation}

\subsection*{Isolation}
Wird über das Betriebssystem verwaltet. Bestimmte Systemaufrufe erlauben das Durchbrechen der Isolation, so dass Prozesse untereinander kommunizieren bzw. miteinander agieren können.

\subsection*{``isoliert``}
Ein isolierter Prozess kann nicht auf den Adressraum eines anderen Prozesses zugreifen.
