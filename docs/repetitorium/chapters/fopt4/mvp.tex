\section{MVP}

\textbf{MVP} gehört zu den \textbf{MV*-Architekturmustern}.\\

\noindent
\textbf{Architekturmuster} besitzen wie \textbf{Entwurfsmuster} einen hohen Abstraktionsgrad.\\

\textbf{Entwurfsmuster} werden i.d.R. zur Lösung kleiner Teilbereiche von Software eingesetzt, während \code{Arhictekturmuster} für die gesamte Software oder für größere Teile einer Software eingesetzt werden\footnote{vgl. Skript FOPT4, S. 1}.\\
$\rightarrow$ Enturfsmuster beschreiben eher das Zusammenwirken von Klassen und Schnittstellen.
Architekturmuster sind wie ein Bauplan für ganze Programmteile bzw. Teilbereiche einer Software zu verstehen.

\noindent
zu den Vorteilen von Architekturmustern gehört:

\begin{itemize}
    \item sie helfen, Anwendungen klarer zu strukturieren $\rightarrow$ Leitfaden zur Strukturierung
    \item erleichtert das systematische Testen (durch klare Strukturierung); Änderungen an der Software sind leichter durchführbar
    \item hift, den Code anderer zu verstehen, wenn dieser nach einem bestimmten Muster strukturiert ist
\end{itemize}

\subsection{Prinzip von MVP}

\textbf{MVP} bezeichnet drei Komponenten: \textbf{Model}, \textbf{View}, \textbf{Presenter}.\\

\subsection*{Model}

Das \textbf{Model} ist verantwortlich für
\begin{itemize}
    \item die \textbf{Geschäftslogik}, die den logischen Kern der Anwendung darstellt
    \item die Bereitstellung und das Ändern relevanter Daten für die Anwendung
    \item die Wahrung von Konsistenzbedingungen
    \item die Realisiserung von Persistierung
\end{itemize}

Das Model ist unabhängig von Darstellungs- und Steuerungslogik.

\subsection*{View}

Die \textbf{View} ist die Darstellungskomponente der Software und baut und verändert sie.\\
Mehrere Views können hierbei Teile des Models darstellen.\\
Benutzerinteraktionen werden von der View an den Presenter weitergeleitet.\\

\subsection*{Presenter}
Die \textbf{Präsentationskomponente} ist zuständig für die Ablaufsteuerung, i.d.R. durch vom Bentzer ausgelöste Aktionen, wie bspw. einem Button-click.

\begin{tcolorbox}
Der \textbf{Presenter} hat eine Vermittlerrolle: Er kennt das \textbf{Model} und kann bei Bedarf vermitteln.\\
Im Unterschied zu \textbf{MVC} kennen sich Model und \textbf{View} nicht.\\
\end{tcolorbox}

Der Presenter kann einen Teil der Validierungslogik enthalten, bspw. zum Erkennen fehlerhafter Eingabedaten.\\

\noindent
Alle Komponenten sind gut testbar:

\begin{itemize}
    \item Das \textbf{Model} hat keien Abhängigkeiten zu \textbf{Presenter} und \textbf{View}.
    \item Der \textbf{Presenter} kann über \textit{Mock-Objekte} (für View & Models) getestet werden
    \item Die \textbf{View} erhält für den Presenter ein Mock-Objekt.
\end{itemize}

\section{Threads und JavaFX}

Stage-Objekte dürfen nciht im \textbf{Main-Thread} erzeugt werden. \\
Das primäre Stage-Objekt wird von der Platform vorgegeben, nachfolgenden Stage-Objekte und UI -Elemente müssen im \textbf{JavaFX Application Thread} erzeugt werden.\\

\noindent
Der \textbf{JavaFX Application Thread} ist auch zuständig für die Ereignisbehandlung - insgesamt gibt es nur einen solchen Thread für die Verwaltung von UI-Elementen und deren Ereignisse. \\

\noindent
Mit der Methode \code{Platform.runLater()} lassen sich Aktionen am \textbf{JavaFX Application Thread} \textit{anmelden} bzw. in dessen \textit{Warteschlange} einreihen (Parameter vom Typ \code{Runnable}).\\

\subsection*{Regel 1}
Die Durchführung der Ereignisbehandlung sollte in \underline{möglichst kurzer Zeit} abgeschlossen sein, auf \code{sleep()} und \code{wait()} sollte möglichst verzichtet werden.

\subsection*{Regel 2}
Der Zugriff auf grafische Elemente der Oberfläche darf \underline{ausschließlich} durch den \textbf{JavaFX Application Thread} erfolgen.\\
Wenn ein anderer Thread auf die Oberfläche zugreifen möchte, beauftragt der dazu den \textbf{JavaFX Application Thread} durch den Aufruf der Methode \code{Platform.runLater()}.\\
Der Auftrag muss in Form eines \code{Runnable}-Objektes übergeben werden.\\

\noindent
Um Theads, die mit der UI interagieren, mit Beendigung des Hauptprogramms zu beenden, sollte man sie als Daemon-Thread deklarieren\footnote{
``Ein Java-Programm ist zu Ende, wenn alle Threads zu Ende sind - Hintergrund-Threads werden dabei nciht beachtet, nur Vordergrund-Threads (vgl.~\cite[89]{Oec22}).
} (Aufruf der methode \code{setDaemon(true)} auf ein Objekt der Klasse \code{java.lang.Thread}, bevor der Thread gestartet wurde).\\

\noindent
\code{Runnable}s werden in der Reihenfolge durchgeführt, in denen sie mit \code{Platform.runLater()} angemeldet worden sind.

\subsection{Tasks und Services in JavaFX}

JavaFX bietet eine Unterstützung zur Verarbeitung länger andauernder / immer wiederkehrender Aktivitäten im Zusammenhanf mit der UI an, über die generische Schnittstelle \code{Worker} sowie der generischen Klassen \code{Task}, \code{Service}, \code{ScheduledService}.\\

\noindent
\textbf{Task} ermöglicht eine einmalige Ausführung einer länger andauernden Aktivität.\\

\noindent
\texbf{Service} ermöglicht die wiederholte Aktivierung, indem immer wieder Task-Objekte erzeugt werden.\\

\noindent
\textbf{ScheduledService} ermöglicht die periodische ADurchführung von Aktivitäten.\\

\noindent
Properties ermöglichen die Kommunikation zwischen \textbf{Tasks}/\code{Services} und dem \textbf{JavaFX Application Thread}, über sie kommt man sogar ohne \code{Platform.runLater()} aus (der gleichzeitige Gebrauch wird aber nicht ausgeschlossen.)\\

\noindent
Die Methode der Listener, die an diese Properties angemeldet sind, werden immer vom \textbf{JavaFX Application Thread} ausgeführt.\\

\noindent
Beispiele für Properties der Klasse \code{Task}\footnote{
Class Task<V>: \url{https://docs.oracle.com/javase/8/javafx/api/javafx/concurrent/Task.html} - abgerufen 29.01.2024
}:

\begin{itemize}
    \item \code{message}: Meldungen von Threads aus dem \textbf{JavaFX Application Thread; die Threads können über \code{updateMessage()}} den Inhalt ändern
    \item \code{title:} wie \code{message} (analoge Funktion: \code{setTitle()})
    \item \code{value}: ähnlich \code{message} und \code{title} (Methode: \code{updateValue()}); der Typ ist \code{ReadOnlyObjectProperty<V>}, wobei \code{V} der generische Typ der Klasse \code{Task} ist.
    \item \code{workDone} / \code{totalWork}: Indikator zum Fortschritt (Relation geleistete Arbeit zur gesamten Arbeit).
    \code{updateProgress()} ermöglicht die Aktualisierung beider Werte auf einmal
    \item \code{progress}: nur lesender Zugriff, liefert immer $\frac{workDone}{totalWork}$, also den Quotienten
    \item \code{state}: liefert den zustand des Tasks (\code{READY}, \code{RUNNING}, \code{SUCCEEDED}, \code{FAILED}, \code{CANCELLED})
\end{itemize}

Um \code{Task} zu verwenden, leitet man eine eigene Klasse von \code{Task} ab und überschreibt \code{call}; in der Methode lässt sich \code{updateValue} und \code{updateProgress} aufrufen (der Rückgabetyp von \code{call} its der Typparameter der KLasse \code{Task}; für \code{extends Task<Double>} also \code{Doubel}).\\
\code{call} sollte überprüfen, ob \code{cancel} aufgerufen wurde (\code{isCancelled}), um die Bearbeitung abzubrechen (im Gegensatz zum \code{interrupt}-Flag bei Threads ist der Zustand des \clode{cancel}-Flags aber persistent).\\

\noindent
\code{Service}s sind für wiederholte Ausführungen länger andauernde Aktivitäten vorgesehen: Services erzeugen hierfür bei jeder wiederholten Ausführung ein neues \code{Task}-Objekt; ein Service kann mittels \code{start()} direkt gestartet werden, ein \code{Task} implementiert die \code{Runnable}-Schnittstelle und wird einem Thread übergeben, der gestartet werden muss.

\begin{minted}[mathescape,
    linenos,
    numbersep=5pt,
    gobble=2,
    frame=lines,
    framesep=2mm]{java}
    MyTask t = new MyTask();
    Thread t1 = new Thread(t);
    t1.setDaemon(true);
    t1.start();
\end{minted}\\

\noindent
Eine Klasse muss von der abstrakten Klasse \code{Service} abbleiten und darin \code{createTask()} überschreiben, die das entsprechende Task-Objekt zurückliefert.\\

\noindent
Man kann dem Service einen \code{Executor} zuweisen, ansonsten wird ein neuer \code{Daemon}-Thread erzeugt und gestartet.\\

\noindent
Erneutes Starten des Service ist erst möglich, wenn der Service nicht mehr aktiv ist (\code{isRunning}), oder wenn zuvor \code{reset()} aufegrufen wurde.\\
Auch \code{restart()} ermöglicht einen Neustart (ruft intern \code{cancel()} $\rightarrow$ \code{reset()} $\rightarrow$ \code{start()} auf).