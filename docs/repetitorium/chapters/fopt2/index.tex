
\chapter{FOPT2 - Fortgeschrittene Synchronisationskonzepte in Java (Parallele Programmierung 2)}

\section{Lehrstoff}

Das Skript FOPT2 bezieht sich auf folgende Inhalte im Buch:

\begin{tcolorbox}[colback=white!20,color=white]
    \begin{enumerate}
        \setcounter{enumi}{8}
        \item \textbf{Semaphore}
        \begin{itemize}
            \item[] Kapitel 3
            \item[] Abschnitt 3.1
        \end{itemize}

        \item \textbf{Message Queues und Pipes}
        \begin{itemize}
            \item[] Abschnitt 3.2
            \item[] Abschnitt 3.3
        \end{itemize}

        \item \textbf{Weitere Synchronisationsprobleme und deren Lösung}
        \begin{itemize}
            \item[] Abschnitt 3.4
            \item[] Abschnitt 3.5
            \item[] Abschnitt 3.6
        \end{itemize}

        \item \textbf{Die concurrent Klassenbibliothek}
        \begin{itemize}
            \item[] Abschnitt 3.7
        \end{itemize}

        \item \textbf{Das Fork-Join-Framework}
        \begin{itemize}
            \item[] Abschnitt 3.8
            \item[] optional: 3.9, 3.10
        \end{itemize}

        \item \textbf{Verklemmungen}
        \begin{itemize}
            \item[] Abschnitt 3.11
            \item[] Abschnitt 3.12
            \item[] Abschnitt 3.13
        \end{itemize}

    \end{enumerate}
\end{tcolorbox}

\newpage


\section{Fortgeschrittene Synchronisationskonzepte in Java}


\include{chapters/fopt2/1.1 Semaphore}


\subsection{Notizen zu den Übungen}

Abfragen auf Listen/Felder von Threads, die sich derzeitig in der Warteschlange befinden, sollte von außen immer \code{synchronized} erfolgen.
Außerdem sollte eine Kopie der Listen zurückgegeben werden, um Manipulation an der tatsächlichen Liste zu verhindern.\\

\noindent
Bei Exceptions, die nicht abgefangen werden, sorgt \code{finally} für ein Ausführen des durch \textit{finally} eingeleiteten Anweisungsblock auch im Fall einer Exception\footnote{
Java Language Specification - 14.20.2. Execution of try-finally and try-catch-finally : \url{https://docs.oracle.com/javase/specs/jls/se21/html/jls-14.html#jls-14.20.2} - abgerufen 26.01.2024
}, wie folgendes Beispiel zeigt:

\newpage
\begin{minted}[mathescape,
    linenos,
    numbersep=5pt,
    gobble=2,
    fontsize=\small,
    frame=lines,
    framesep=2mm]{java}
    public class TryCatchDemo {

        public String m1(boolean exc) {
            try {

                System.out.println("try 1.");

                if (exc) {
                    throw new Exception();
                }
                System.out.println("try 2.");

                return "foo";

            } catch (Exception e) {

                return "Exception.";

            } finally {
                return "finally.";
            }
        }

        public static void main(String[] args) {
           TryCatchDemo demo = new TryCatchDemo();
           System.out.println(demo.m1(false));
           System.out.println();
           System.out.println(demo.m1(true));
        }

    }
\end{minted}\\

\noindent
Die Ausgabe des Programms lautet:


\noindent
\begin{minted}[mathescape,
    numbersep=5pt,
    gobble=2,
    frame=none,
    framesep=2mm]{bash}
    try 1.
    try 2.
    finally.
    foo

    try 1.
    finally.
    Exception.
\end{minted}\\

\newpage

\include{chapters/fopt2/1.2 Message Queues}


\include{chapters/fopt2/1.3 Concurrent Klassenbibliothek}

\include{chapters/fopt2/1.4 Verklemmungen}





\newpage
\section*{Notizen}

\newpage
