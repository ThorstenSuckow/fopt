\section{Fortgeschrittene Synchronisationskonzepte in Java}
\subsection{Semaphore}

Der \textbf{Semaphor} (plural: \textbf{Semaphore}; engl. \textit{semaphore}) repräsentiert ein klassisches Synchronisationskonzept.\\

\noindent
Das abstrakte Konzept dahinter: Ein Semaphor hat einen Wert des Typs \code{int}, der nie $<\ 0$ werden kann.\\
Die Methoden \code{p()} (nl.: \textit{passeeren}, auch engl. \textit{down} oder \textit{acquire}) bzw. \code{v()} (nl. \textit{frijgeven}, auch engl. \textit{up} oder \textit{release}) dienen zum Herunter- bzw. Hochzählen des Attributs.\\

\noindent
Ein Thread wird \textbf{blockiert}, wenn bei seinem Aufruf von \code{p()} der Attributwert negativ werden würde.

\subsection{Einfache Semaphore für den gegenseitigen Ausschluss}

Semaphore werden häufig zur Realisierung des gegenseitigen Ausschlusses realisiert $\rightarrow$ ein Programmstück kann nur von einem Thread gleichzeitig ausgeführt werden.\\

\noindent
Hierzu werden sogenannte \textbf{Mutex Semaphors} (\textit{Mutex} = \textit{Mutual Exclusion}) verwendet:\\
Der Semaphor wird hierzu mit dem Wert $1$ initialisiert - ruft ein Thread $t_1$ \code{p()} auf und zählt dabei den Wert des Semaphors herunter, müssen ankommende Threads $t$ warten, bis der Wert durch die Freigabe von $t_1$ wieder erhöht und so der kritische Bereich\footnote{
    \textit{kritischer Bereich}: das Programmstück, das zu einem Zeitpunkt nur von höchstens einem Thread ausgeführt werden darf (vgl.~\cite[102]{Oec22}).
} wieder freigegeben wird\footnote{ein \textit{binärer Semaphor} kann nur die Werte $0$ und $1$ bzw. $true$ und $false$ annehmen. In \textit{v()} kann vor dem \textit{notify()} überprüft werden, ob der Semaphor gerade das Objekt gesperrt hat}.\\

\noindent
In dem folgenden Beispiel wird \underlien{ein} Semaphor für 3 verschiedene Objekte desselben Typs verwendet.\\
Der Semaphor stellt sicher, dass nur jeweils ein Thread das Programmstück in \code{enter()} (über die $3$ verschiedenen Objekte verteilt) ausführen kann. \\

\newpage
\begin{minted}[mathescape,
    linenos,
    numbersep=5pt,
    gobble=2,
    fontsize=\small,
    frame=lines,
    framesep=2mm]{java}
    public class MutexSemaphorDemo {

        static class MutexSemaphor {
            private int n = 1;
            public synchronized void p() {
                while (n < 1) {
                    try {
                        wait();
                    } catch (InterruptedException ignored) {}
                }
                n--;
            }

            public synchronized void v() {
                n++;
                notify();
            }
        }

        static class Waiter extends Thread{
            MutexSemaphor sem;
            String name;
            public Waiter(MutexSemaphor s, String n) {
                sem = s;
                name = n;
                start();
            }

            public void enter() {
                    String currentThread = Thread.currentThread().getName();
                    sem.p();
                    System.out.println(currentThread + " enters [" + name + "]!");
                    try {
                        Thread.currentThread().sleep((int) (Math.random() * 100));
                    } catch (InterruptedException ignored) {}
                    System.out.println(currentThread + " exits [" + name + "]");
                    sem.v();
            }
        }

        public static void main(String[] args) throws InterruptedException {
            MutexSemaphor sem = new MutexSemaphor();
            Waiter waiter1 = new Waiter(sem, "w1");
            Waiter waiter2 = new Waiter(sem, "w2");
            Waiter waiter3 = new Waiter(sem, "w3");
            Thread t1 = new Thread(waiter1::enter);
            Thread t2 = new Thread(waiter2::enter);
            Thread t3 = new Thread(waiter3::enter);

            t1.start();t2.start();t3.start();
            t1.join();t2.join();t3.join();
        }
    }
\end{minted}\\

\newpage
Eine mögliche Ausgabe für das Programm ist:\\

\noindent
\begin{minted}[mathescape,
    numbersep=5pt,
    gobble=2,
    frame=none,
    framesep=2mm]{bash}
    Thread-3 enters critical section of [w1]!
    Thread-3 exits [w1]
    Thread-4 enters critical section of [w2]!
    Thread-4 exits [w2]
    Thread-5 enters critical section of [w3]!
    Thread-5 exits [w3]
\end{minted}\\

\noindent
Wäre \code{enter()} hingegen bloß \code{synchronized} und es würde kein Semaphor verwendet, würden die Threads jeweils parallel \code{enter()} ausführen, die Ausgabe lautet dann in etwa:


\begin{minted}[mathescape,
    numbersep=5pt,
    gobble=2,
    frame=none,
    framesep=2mm]{bash}
    Thread-3 enters critical section of [w1]!
    Thread-4 enters critical section of [w2]!
    Thread-5 enters critical section of [w3]!
    Thread-4 exits [w2]
    Thread-3 exits [w1]
    Thread-5 exits [w3]
\end{minted}\\

\subsection{Semaphore zur Herstellung vorgegebener Ausführungsreihenfolgen}

Neben gegenseitigem Ausschluss können Semaphore auch eingesetzt werden, um die \textbf{Ausführungsreihenfolge} von Threads festzulegen\footnote{ausführliches Beispiel in~\cite[104]{Oec22}}.

\subsection{Additive Semaphore}
\textbf{Additive Semaphore} erlauben es aufrufenden Threads, den Wert eines Semaphors um ein gewisses $\Delta\ (\geq\ 1)$ herunterzuzählen.\\

\noindent
Dabei ändert sich die Wartebedingung des Semaphors von

\begin{minted}[mathescape,
    numbersep=5pt,
    gobble=2,
    frame=none,
    framesep=2mm]{java}
    while (n == 0) {
        //...
    }
\end{minted}\\

zu

\begin{minted}[mathescape,
    numbersep=5pt,
    gobble=2,
    frame=none,
    framesep=2mm]{java}
    while (n - x < 0) {
        //...
    }
\end{minted}\\

\noindent
Durch die parametrisierte Wertebedingung ist es nötig, wartende Threads mittels \code{notifyAll()} aus der Warteschlange zu holen: Erstens kann es mehrere Threads geben, bei denen insgesamt $\sum_{i=1}^{n} \Delta_i\ \geq\ 0$ sein kann (es dürfen dann mehrere Threads parallel in den kritischen Bereich); zweitens kann die Wartebedingung basierend auf dem $x$ für Threads unterschiedlich sein - sollte nur \code{notify} verwendet werden kann es passieren, dass ein Thread direkt danach wieder in die Warteschlange kommt - und infolgedessen kein anderer Thread mehr in den kritischen Bereich, und kein Aufruf von \code{v()} mehr stattfindet.\\

\noindent
Es sollte selbstverständlich sein, dass in einer Implementierung der Wert des Semaphors \textbf{auf einen Schlag} (durch Subtraktion / Addition) verändert wird, ansonsten kann es zu einer \textbf{Verklemmungssituation} kommen, wenn der Wert Stückweise inkrementiert/dekrementiert wird (vgl.~\cite[109]{Oec22}).

\subsection{Semaphorgruppen}

Die grundlegende Eigenschaft von additiven Semaphore, den Wert eines Semaphores auf einen Schlag zu ändern, ist Motivation für die Einführung von \textbf{Semaphorgruppen}.\\

\noindent
Semaphorgruppen werden nicht als Felder von Semaphore angelegt, sondern eine Semaphorgruppe enthält ein Feld von Werten, auf dem operiert wird.\\

\noindent
Jeder Eintrag in dem Feld repräsentiert ein Mitglied der Semaphorgruppe.


\noindent
Auf die Implementierung von \code{p()}/\code{v()} wird hierbei verzichtet - stattdessen gibt es eine Methode mit einer Signatur ähnlich zu

\begin{minted}[mathescape,
    numbersep=5pt,
    gobble=2,
    frame=none,
    framesep=2mm]{java}
    public synchronized void changeValues(int[] deltas)
\end{minted}\\

\noindent
Die Methode schiebt so lange einen Thread in eine Warteschlange, bis die Wartebedingung nicht mehr erfüllt ist - hierbei wird dann jeder Index von der Semaphore wie folgt überprüft:


\begin{minted}[mathescape,
    numbersep=5pt,
    gobble=2,
    frame=none,
    framesep=2mm]{java}
    if (values[i] + delta[i] < 0) {
        return false
    }
\end{minted}\\

\noindent
Änderungen werden dann nur durchgeführt, wenn nach der Änderung die Werte aller Semaphore der Gruppe nicht negativ sind (vgl.~\cite[109]{Oec22}).\\

\noindent
Wie bei additiven Semaphore ist es nötig, Threads über \code{notifyAll()} aus der Warteschlange zu holen - ansonsten kann es auch hier vorkommen, dass ein Thread, der weiterlaufen könnte, nicht geweckt wird.\\

\noindent
Das folgende Beispiel zeigt eine Implementierung einer einfachen Semaphorgruppe (vgl.~\cite[110, Listing 3.5]{Oec22}).
Auf die Überprüfung der Argumente in den Methoden (Eingabelänge == Argumentlänge) wurde verzichtet.

\newpage
\begin{minted}[mathescape,
    linenos,
    numbersep=5pt,
    gobble=2,
    fontsize=\small,
    frame=lines,
    framesep=2mm]{java}
    class SemaphoreGroup {
        int[] values;
        public SemaphoreGroup (int n) {
            values = new int[n];
            for (int i = 0; i < n; i++) {
                values[i] = 1;
            }
        }

        public synchronized void change(int[] set) {

            while (!canChange(set)) {
                try {
                    wait();
                } catch (InterruptedException ignored) {}
            }

            changeValues(set);
            notifyAll();
        }

        private void changeValues(int[] set) {
            for (int i = 0; i < set.length; i++) {
                values[i] =  values[i] + set[i];
            }
        }

        private boolean canChange(int[] set) {
            for (int i = 0; i < set.length; i++) {
                if (values[i] + set[i] < 0) {
                    return false;
                }
            }
            return true;
        }

    }
\end{minted}
