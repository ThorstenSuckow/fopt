\subsection{Concurrent Klassenbibliothek aus Java 5}

Vor Java 5 gab es im wesentlichen zur Realisierung von Parallelität nur die Klasse \code{Thread}, zur Realisierung von \textbf{Synchronisation} im Wesentlichen nur \code{synchronized, wait, notify und notifyAll}.\\

\noindent
Seit Java 5 gibt es drei weitere Packages, die Klassen und Schnittstellen anbieten, die mit Synchronisation und Parallelität zu tun haben:



\begin{minted}[mathescape,
    numbersep=5pt,
    gobble=2,
    frame=none,
    framesep=2mm]{java}
    java.util.concurrent
    java.util.concurrent.atomic
    java.util.concurrent.locks
\end{minted}\\


Die \code{concurrent}-Klassenbibliothek lässt sich in 5 Themenbereiche aufteilen:

\subsection*{Executor- und ExecutorService}
Mit Hilfe der Schnittstellen \code{Executor} und \code{ExecutorService} kann man Aufträge erteilen, die asynchron ausgeführt werden.\\

\noindent
\code{ExecutorService} enthält Methoden, mit denen man das Ergebnis erteilter Aufträge über ein \code{Future}-Objekt (später) abholen kann.\\

\noindent
Eine wichtige Klassen, die die \code{ExecutorSchnittstelle} implementiert, ist die Klasse \code{ThreadPool}: Mit ihr kann es es sich sparen, selbst Threads erzeugen und starten zu müssen (s. a. \cite[146]{Oec22}).\\

\noindent
\code{Lock}s und \code{Condition}s sind eine Alternative für \code{synchronized}, \code{wait},\code{notify} und \code{notifyAll}.\\

\noindent
\code{Lock} stellt \code{lock} und \code{unlock}\footnote{
    \textit{unlock()} sollte stets in einem \textit{finally}-Block aufgerufen werden (vgl.~\cite[150]{Oec22})
} zum Sperren/Entsperren eines Objektes zur Verfügung.

\noindent
\code{Condition}s sind Objekte, die man sich von einem \code{Lock}-Objekt geben lassen kann und die mit diesem \code{Lock}-Objekt assoziiert sind.\\
Auf \code{Condition}s kann man mit \code{await()} warten\\
\$rightarrow\ wie bei \code{wait()} sorgt \code{await()} dafür, dass der Thread dabei blockiert wird und die Sperre auf den mit der \code{Condition} assoziierten \code{Lock} freigegeben wird.\\
\code{signal()} bzw. \code{signalAll()} sorgen dafür, dass durch eine \code{Condition} blockierte Threads geweckt werden.\\

\noindent
Man kann sich zu eine \code{Lock} beliebig viele \code{Condition}s geben lassen, was eine sehr fein granulierte Modellierung der Bedingung(en) ermöglicht.\\

\noindent
Die \code{Atomic}-Klassen bieten eine Objekthülle für verschiedene Datentypen (\code{Boolean}, \code{Integer}, \code{Long}...) und ermöglichen einen Thread-sicheren lesenden/schreibenden Zugriff auf die umhüllten Werte.\\

\noindent
Mit den \code{Atomic}-Klassen lassen sich auch \code{lock-free} Synchronisation realisieren, die zwar eine Form des aktiven Wertes darstellen, unter Umständen aber effizienter sein können, als eine durch eine Sperre erfolgte Umschaltung auf einen andern Thread.\\

\noindent
Zu den \textbf{Synchronisationsklassen} zählen die Klassen
\begin{itemize}
    \item \code{Semaphore}: Hier entspricht \code{acquire()} dem \code{p()} und \code{release()} dem \code{v()}.
    \item \code{CountdownLatch}: Zähler, der heruntergezählt wird; Threads warten mittels \code{await()}, bis der Zähler $0$ wird.
    \item \code{CyclicBarrier}: $n$ Threads warten gegenseitig aufeinander.
    \item \code{Exchanger}: Ähnlich \code{CyclicBarrier}, nur warten zwei Threads aufeinander und tauschen parametrisiert (\code{Exchanger<V>}) Daten untereinander aus.
\end{itemize}\\

\noindent
\code{Queue}-Klassen ermöglichen den \code{synchronisierten} Datenaustausch zwischen Thread nach dem Erzeuger-Verbraucher-Prinzip: Ein Objekt in eine volle Warteschlange zu legen wird so lange blockiert, bis die Warteschlange nicht mehr voll ist; entsprechend wird der Versuch blockiert, aus einer leeren Warteschlange etwas zu entnehmen (vgl.~\cite[164]{Oec22}).

\subsection{Das Fork-Join-Framework von Java 7}

Die \code{concurrent}-Bibliothek wurde in Java 7 um das \textbf{Fork-Join-Framework} erweitert.\\

\noindent
Eine zentrale Klasse davon ist \code{ForkJoinPool}, der wie der \code{ThreadPoolExecutor} ein \code{ThreadPool} realisiert und \code{ExecutorService} implementiert.\\

\noindent
\$rightarrow\ dieser \code{ThreadPool} ist speziell für baumartige Berechnungen gedacht.

\begin{tcolorbox}[enlarge top by=0.5cm,enlarge bottom by=0.5cm]
    Ein Objekt von \code{RecursiveTask} / \code{RecursiveAction} stellt einen Knoten des Berechnungsbaumes dar.
\end{tcolorbox}\\

\noindent
Der Konstruktor von \code{ForkJoinPool} erlaubt auch Übergabe eines \code{int}-Wertes, mit dem man den Parallelitätsgrad angeben kann\footnote{default entspricht der Anzahl der Prozessoren}.\\

\noindent
Aufträge werden durch Objekte der generischen Klasse \code{ForkJoinTask} repräsentiert, wobei der Typparameter der Typ des Resultats ist.\\

\noindent
\code{RecursiveAction}/\code{RecursiveTask} sind aus \code{ForkJoinTask} abgeleitet: \code{RecursiveAction} ist für Aufträge ohne Resultat gedacht (bspw. sortieren), \code{RecursiveTask} für Aufträge mit Ergebnis:\\

\noindent
Ein Thread eines \code{ForkJoinPools} kann weitere Aufträge bearbeiten, sobald er mit \code{join()} auf das Ende eines anderen Auftrages wartet - er ist durch das Warten also nicht blockiert; dadurch kann ein \code{ForkJoinPool} auch mit wenigen Threads baumartig verzweigte Berechnungen durchführen - das ist mit einem normalen \code{ThreadPool} nicht möglich (vgl.~\cite[168]{Oec22})\footnote{
    ebenda wird dies als der wesentliche Unterschied zwischen \textit{ForkJoinPool} und \textit{ThreadPoolExecutor} bezeichnet.
}.\\

\noindent
Alle Threads des \code{ForkJoinPool}s sind \textbf{Daemon}-Threads.\\

\noindent
Vorgehen bei der Implementierung - \textbf{Divide & Conquer}: Prüfen, ob Berechnung klein genug, um direkt zu bearbeiten, falls nein, neue \code{ForkJoinTasks} erzeugen und durch \code{fork()} dem \textbf{ThreadPool} übergeben, mit \code{join} auf auf die Bearbeitung warten und dann Teilergebnisse zu Gesamtergebnis kombinieren.

Das folgende Beispiel ist \cite[168, Listing 3.25]{Oec22} entnommen und beinhaltet die Implementierung von \code{compute()}\footnote{
    Class RecursiveTask<V> - compute(): \url{https://docs.oracle.com/en/java/javase/21/docs/api/java.base/java/util/concurrent/RecursiveTask.html#compute()} - abgerufen 27.01.2024
} zur Berechnung der Fakultät:
\begin{minted}[mathescape,
    linenos,
    numbersep=5pt,
    gobble=2,
    fontsize=\small,
    frame=lines,
    framesep=2mm]{java}
    class RecursiveTaskImpl extends RecursiveTask<Integer> {

        private int n;

        public RecursiveTaskImpl(int n) {
            this.n = n;
        }

        public Integer compute() {
            if (n == 0) {
                return 1;
            }

            RecursiveTaskImpl t = new RecursiveTaskImpl(n - 1);
            t.fork();
            int r = t.join();
            return n * r;
        }
    }
\end{minted}\\