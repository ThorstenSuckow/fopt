\section{Dienstag}

\subsection*{ToggleGroups}
\code{RadioButton}s lassen sich über eine \code{ToggleGroup}\footnote{
``Class ToggleGroup``: \url{https://docs.oracle.com/javase/8/javafx/api/javafx/scene/control/ToggleGroup.html} - abgerufen 20.2.2024
} gruppieren:

\begin{minted}[mathescape,
    linenos,
    numbersep=5pt,
    gobble=2,
    frame=lines,
    framesep=2mm]{java}
    RadioButton one = new RadioButton("one");
    RadioButton two = new RadioButton("two");
    RadioButton three = new RadioButton("three");

    ToggleGroup radioButtonGroup = new ToggleGroup();
    radioButtonGroup.getToggles().addAll(one, two, three);
\end{minted}\\

\subsection*{StringProperty an IntegerProperty binden}

\begin{center}\code{Bindings.convert(observableValue: ObservableValue<?>): StringExpression}\end{center}
ermöglicht es, bsp. ein JavaFX-\code{Label} an eine \code{SimpleIntegerProperty} zu binden\footnote{
``convert``: \url{https://docs.oracle.com/javase/8/javafx/api/javafx/beans/binding/Bindings.html#convert-javafx.beans.value.ObservableValue-} - abgerufen 20.2.2024
}.\\

\noindent
Das Folgende Beispiel bindet 3 Labels an drei verschiedene \code{SimpleIntegerProperty}s.\\
Die Summe dieser Properties wird in dem Label code{totalLabel} angezeigt:
\begin{minted}[mathescape,
    linenos,
    numbersep=5pt,
    gobble=2,
    frame=lines,
    framesep=2mm]{java}
    Label lineCountLabel = new Label();
    Label rectangleCountLabel = new Label();
    Label circleCountLabel = new Label();
    Label totalLabel = new Label();

    lineCountLabel.textProperty().bind(Bindings.convert(lineCountProperty));
    rectangleCountLabel.textProperty().bind(Bindings.convert(rectangleCountProperty));
    circleCountLabel.textProperty().bind(Bindings.convert(circleCountProperty));

    totalLabel.textProperty().bind(
        Bindings.convert(lineCountProperty.add(rectangleCountProperty).add(circleCountProperty))
    );

\end{minted}\\