\chapter{WS13}\label{ch:klausurws13}


\subsection{Lösungsvorschlag}


\begin{minted}[mathescape,
    linenos,
    numbersep=5pt,
    gobble=2,
    fontsize=\small,
    frame=lines,
    framesep=2mm]{java}
    class Zahlenschloss {

        private int[] kombination;

        private int[] state;

        private boolean opened = false;

        public Zahlenschloss(int[] kombination) {
            this.kombination = kombination;
            this.state = new int[kombination.length];
        }

        public int anzahlRaedchen() {
            return kombination.length;
        }

        public synchronized int lesen(int radnummer) {
            return state[radnummer];
        }

        public synchronized void drehen(int radnummer, int zahl) {

            state[radnummer] = zahl;
            opened = true;
            for (int i = 0; i < anzahlRaedchen(); i++) {
                if (lesen(i) != kombination[i]) {
                    opened = false;
                    break;
                }
            }

            if (opened) {
                this.notify();
            }
        }

        public synchronized void warten() {

            while (!opened) {
                try {
                    this.wait();
                } catch (InterruptedException ignored) {}
            }
        }
    }
\end{minted}\\


\subsection{Anmerkung und Ergänzungen}

\begin{itemize}
    \item Es wird eine Wartebedingung benötigt, und zwar für die Methode \code{warten()}; ankommende Threads werden
    in die Warteschlange des Zahlenschloss-Objektes geschickt, wenn \code{opened} auf false gesetzt ist, ansonsten
    verlassen diese direkt die Methode wieder.\\
    Die Methode \code{drehen} benötigt keine separate Wartebedingung.
    Es reicht aus, sicherzustellen, dass das Zahlenschloss nicht gleichzeitig von anderen Threads benutzt werden kann:
    Die Methode \code{drehen} ist hierfür synchronisiert, damit das Zahlenschloss {insg.} immer nur eine Zustandsänderung
    erfährt - es sind andere Implementierungen möglich, in denen das Zahlenschloss dann von mehreren Threads gleichzeitig
    genutzt werden darf, wenn sich die Zugriffe anhand der ``Ziel``-\code{radnummer} unterscheiden, {bspw.} durch Mutex-Semaphore,
    die pro Radnummer verwendet werden\footnote{
        der gleichzeitige Zugriff auf unterschidliche Arrays-Indizes ist erlaubt, s. `´17.4.1. Shared Variables``: \url{https://docs.oracle.com/javase/specs/jls/se21/html/jls-17.html#jls-1.4.1} - abgerufen 14.2.2024
    }.
    \item Es gibt nur eine Wartebedingung, von daher sollte \code{notify()} genügen.\\
    Wenn wir allerdings davon ausgehen, dass mehrere Threads über die Methode \code{warten()} in die Warteschlange des Objektes eingereiht worden sind,  sollte \code{notifyAll()} verwendet werden (siehe hierzu auch Abschnitt \ref{subsec:notifyAll}).
    Dennoch ist nicht garantiert, dass auch alle Threads aus der Warteschlange gelangen, denn es kann sein, dass ein anderer Thread die Methode \code{drehen()} betritt, dort die
    Zahlenkombination ändert und \code{opened} wieder auf \code{false} gesetzt wird. \\
    Ein anderer Thread, der nun in  \code{warten()} an die Reihe kommt, überprüft die Wartebedingung, und wird wieder in die Warteschlange eingereiht.
    Es ist also durchaus möglich, dass ein Thread nicht mehr aus der Methode \code{warten()} herauskommt.\\
    Dies könnte bspw. dadurch verhindert werden, dass die Threads in eine Queue gepackt werden, und in \code{drehen()} eine Wartebedingung eingefügt wird, die erst erfüllt ist,
    wenn die Queue geleert wurde oder aus ihr entnommen wurde, in der Reihenfolge, in der die Threads in die Queue eingereiht worden sind (\textit{FIFO}) (s. a. Abschnitt~\ref{subsec:readerwriterproblem}).
    \item Bei der Teilaufgabe mit der Schleife muss die komplette Schleife synchronisiert werden, was man durch ein \code{synchronized}-Statement erreicht\footnote{siehe Abschnitt~\ref{subsec:synchronizedstatement}.}
    \begin{minted}[mathescape,
        linenos,
        numbersep=5pt,
        gobble=2,
        fontsize=\small,
        frame=lines,
        framesep=2mm]{java}
        synchronized (zk) {
            for (int i = 0; i < anzahlRaedchen; i++) {
                System.out.println(zk.lesen(i));
            }
        }
    \end{minted}
    Ansonsten läuft man Gefahr, dass sich nach Auslesen der 1. Position der Wert von Position 2 geändert hat und dadurch eine
    Zahlenkombination ausgegeben wird, die es nicht gegeben hat:
    \begin{enumerate}
        \item $K\coloneqq[0, 0, 0]$
        \item Position $K_0$ wird ausgelesen und liefert $0$.
        \item Thread ändert $K_0$ zu $1$ $\implies K\coloneqq[1, 0, 0] $.
        \item Thread ändert $K_1$ zu $2$ $\implies K\coloneqq[1, 2, 0] $.
        \item Thread ändert $K_2$ zu $3$ $\implies K\coloneqq[1, 2, 3] $.
        \item Positionen $K_1$ und $K_2$ werden ausgelesen und liefern: $2, 3$
        \item Ausgabe: $0, 2, 3$ - diese Kombination hat es in dem Fall aber tatsächlich nicht gegeben.
    \end{enumerate}
\end{itemize}
