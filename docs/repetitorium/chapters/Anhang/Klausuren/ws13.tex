\chapter{WS13}\label{ch:klausurws13}

\section{Aufgabe 1}
\subsection{Lösungsvorschlag}


\begin{minted}[mathescape,
    linenos,
    numbersep=5pt,
    gobble=2,
    fontsize=\small,
    frame=lines,
    framesep=2mm]{java}
    class Zahlenschloss {

        private int[] kombination;

        private int[] state;

        private boolean opened = false;

        public Zahlenschloss(int[] kombination) {
            this.kombination = kombination;
            this.state = new int[kombination.length];
        }

        public int anzahlRaedchen() {
            return kombination.length;
        }

        public synchronized int lesen(int radnummer) {
            return state[radnummer];
        }

        public synchronized void drehen(int radnummer, int zahl) {

            state[radnummer] = zahl;
            opened = true;
            for (int i = 0; i < anzahlRaedchen(); i++) {
                if (lesen(i) != kombination[i]) {
                    opened = false;
                    break;
                }
            }

            if (opened) {
                this.notify();
            }
        }

        public synchronized void warten() {

            while (!opened) {
                try {
                    this.wait();
                } catch (InterruptedException ignored) {}
            }
        }
    }
\end{minted}\\


\subsection{Anmerkung und Ergänzungen}

\begin{itemize}
    \item Es wird eine Wartebedingung benötigt, und zwar für die Methode \code{warten()}; ankommende Threads werden
    in die Warteschlange des Zahlenschloss-Objektes geschickt, wenn \code{opened} auf false gesetzt ist, ansonsten
    verlassen diese direkt die Methode wieder.\\
    Die Methode \code{drehen} benötigt keine separate Wartebedingung.
    Es reicht aus, sicherzustellen, dass das Zahlenschloss nicht gleichzeitig von anderen Threads benutzt werden kann:
    Die Methode \code{drehen} ist hierfür synchronisiert, damit das Zahlenschloss {insg.} immer nur eine Zustandsänderung
    erfährt - es sind andere Implementierungen möglich, in denen das Zahlenschloss dann von mehreren Threads gleichzeitig
    genutzt werden darf, wenn sich die Zugriffe anhand der ``Ziel``-\code{radnummer} unterscheiden, {bspw.} durch Mutex-Semaphore,
    die pro Radnummer verwendet werden\footnote{
        der gleichzeitige Zugriff auf unterschiedliche Arrays-Indizes ist erlaubt, s. `´17.4.1. Shared Variables``: \url{https://docs.oracle.com/javase/specs/jls/se21/html/jls-17.html#jls-1.4.1} - abgerufen 14.2.2024
    }.
    \item Es gibt nur eine Wartebedingung, von daher sollte \code{notify()} genügen.\\
    Wenn wir allerdings davon ausgehen, dass mehrere Threads über die Methode \code{warten()} in die Warteschlange des Objektes eingereiht worden sind,  sollte \code{notifyAll()} verwendet werden (siehe hierzu auch Abschnitt \ref{subsec:notifyAll}).
    Dennoch ist nicht garantiert, dass auch alle Threads aus der Warteschlange gelangen, denn es kann sein, dass ein anderer Thread die Methode \code{drehen()} betritt, dort die
    Zahlenkombination ändert und \code{opened} wieder auf \code{false} gesetzt wird. \\
    Ein anderer Thread, der nun in  \code{warten()} an die Reihe kommt, überprüft die Wartebedingung, und wird wieder in die Warteschlange eingereiht.
    Es ist also durchaus möglich, dass ein Thread nicht mehr aus der Methode \code{warten()} herauskommt.\\
    Dies könnte bspw. dadurch verhindert werden, dass die Threads in eine Queue gepackt werden, und in \code{drehen()} eine Wartebedingung eingefügt wird, die erst erfüllt ist,
    wenn die Queue geleert wurde oder aus ihr entnommen wurde, in der Reihenfolge, in der die Threads in die Queue eingereiht worden sind (\textit{FIFO}) (s. a. Abschnitt~\ref{subsec:readerwriterproblem}).
    \item Bei der Teilaufgabe mit der Schleife muss die komplette Schleife synchronisiert werden, was man durch ein \code{synchronized}-Statement erreicht\footnote{siehe Abschnitt~\ref{subsec:synchronizedstatement}.}
    \begin{minted}[mathescape,
        linenos,
        numbersep=5pt,
        gobble=2,
        fontsize=\small,
        frame=lines,
        framesep=2mm]{java}
        synchronized (zk) {
            for (int i = 0; i < anzahlRaedchen; i++) {
                System.out.println(zk.lesen(i));
            }
        }
    \end{minted}
    Ansonsten läuft man Gefahr, dass sich nach Auslesen der 1. Position der Wert von Position 2 geändert hat und dadurch eine
    Zahlenkombination ausgegeben wird, die es nicht gegeben hat:
    \begin{enumerate}
        \item $K\coloneqq[0, 0, 0]$
        \item Position $K_0$ wird ausgelesen und liefert $0$.
        \item Thread ändert $K_0$ zu $1$ $\implies K\coloneqq[1, 0, 0] $.
        \item Thread ändert $K_1$ zu $2$ $\implies K\coloneqq[1, 2, 0] $.
        \item Thread ändert $K_2$ zu $3$ $\implies K\coloneqq[1, 2, 3] $.
        \item Positionen $K_1$ und $K_2$ werden ausgelesen und liefern: $2, 3$
        \item Ausgabe: $0, 2, 3$ - diese Kombination hat es in dem Fall aber tatsächlich nicht gegeben.
    \end{enumerate}
\end{itemize}

\begin{tcolorbox}[colback=red!20,color=white,title=Anmerkung]
    Die Methode \code{lesen()} als \code{synchronized} zu markieren könnte man sich vlt. sparen, wenn man davon ausgeht,
    dass die Methode ohnehin in einem \code{synchronized}-Statement verwendet wird, um alle Rädchen abzulesen.\\
    Mehrere Threads können also nicht parallel auf unterschiedliche Positionen des Feldes zugreifen, wenn die Methode
    synchronisiert ist.\\
    Allerdings ist sowohl das Skript als auch das Buch recht klar, was in dieser Situation geschehen muss (s. Skript Fopt1/2, S. 9, außerdem \cite[31, Abschnitt 2.3.6]{Oec22}): Es muss (in diesem Kurs) immer \code{synchronized} verwendet werden, wenn gleichzeitig
    Daten geschrieben und gleichzeitig diese Daten gelesen werden sollen - und eine andere Implementierung, bei der die
    einzelnen Positionen ``gelocked`` sind, so dass ein gleichzeitiger Zugriff auf unterschiedliche Rädchen möglich ist, war nicht gefordert.\\
    Ggfl. würde in anderen Implementierungen der Einsatz von \code{AtomicReferenceArray}\footnote{s. \cite[157 ff.]{Oec22}
    s. ``Class AtomicReferenceArray<E>``: \url{https://docs.oracle.com/en/java/javase/21/docs/api/java.base/java/util/concurrent/atomic/AtomicReferenceArray.html} - abgerufen 15.2.2024
    } Sinn machen, aber das Lehrmaterial ist bereits sehr eindeutig bzgl. der Verwendung von \code{synchronized}.
\end{tcolorbox}



\section{Aufgabe 3}
\subsection{Lösungsvorschlag}

\subsection*{Statische Parallelität}
Statische Parallelität erlaubt es einem Server, eine \textit{fixe} Anzahl von Verbindungen gleichzeitig zu bedienen.\\
Hierbei wird ein Feld von Threads erstellt, wobei jeder Thread das \code{ServerSocket}-Objekt als Referenz übergeben bekommt.
In der \code{run()}-Methode wird dann über \code{accept()} in einer Endlosschleife auf eingehende Verbindungen gewartet, die dann so lange bedient werden, bis sich ein Client wieder abmeldet (oder eine andere Abbruchbedingung erfüllt ist, wie z.B. ein \code{SocketTimeout}).\\
Das sich ein Client abmeldet, bekommt man bspw. dadurch mit, dass \code{null} beim Lesen von einer Nachricht des Clients zurückgegeben wird (vgl. \cite[286]{Oec22}. \\
Siehe Abschnitt~\ref{sec:seqparserver} für ein Implementierungsbeispiel.

\subsection*{Dynamische Parallelität}

Bei \textbf{Dynamischer Parallelität} erzeugt der Server für jede Verbindung einen neuen Thread, der so lange läuft, bis der Client die Verbindung wieder trennt (oder eine andere Abbruchbedingung erfüllt ist).\\
Die Anzahl der Threads ändert sich dadurch laufend.\\
Wird die max. Anzahl erlaubter Threads nicht kontrolliert, kann es zu einer Überlastung des Rechners kommen, auf dem der Server läuft (bspw. durch einen Denial-of-Service-Angriff.)\\
Siehe Abschnitt~\ref{sec:seqparserver} für ein Implementierungsbeispiel.

\noindent
I.d.R. ist eine Mischform aus beidem geeignet, um mehrere Clients gleichzeitig bedienen zu können, und dabei nicht Gefahr zu laufen, durch dynamisches, unbegrenztes Wachstum der Anzahl der Threads überlastet zu werden.\\
Eine feste Vorgabe existierender Threads kann dabei ansonsten benötigte Rechenoperationen zur Instanziierung neuer Thread-Objekte
einsparen.\\
Ein \code{ThreadPoolExecutor}\footnote{
    s. ``Class ThreadPoolExecutor``: \url{https://docs.oracle.com/en/java/javase/21/docs/api/java.base/java/util/concurrent/ThreadPoolExecutor.html} - abgerufen 16.2.2024
} kann verwendet werden, um eine Mischform aus statischer und dynamischer Parallelität zu realisieren (vgl.\cite[164 u. 302]{Oec22})\footnote{
    es sollte dann \textit{maximumPoolSize} $>$ \textit{corePoolSize} gelten und die \textit{BlockingQueue} (s. ``Interface BlockingQueue<E>``: \url{https://docs.oracle.com/en/java/javase/21/docs/api/java.base/java/util/concurrent/BlockingQueue.html} - abgerufen 31.01.2024; s.a.~\cite[146]{Oec22}) sollte eine begrenzte Anzahl von Plätzen haben (darf nicht beliebig anwachsen).
}.

\subsection*{Client sendet Kommandos in einer Schleife, und empfängt Kommandos in einer zweiten}

Der Client $C$ verbindet sich zu dem Server $S$ und sendet in einer Schleife Kommandos an den Server, in einer zweiten
empfängt er die Antworten des Servers.\\
Der Programmcode dazu könnte so aussehen:

\begin{minted}[mathescape,
    linenos,
    numbersep=5pt,
    gobble=2,
    fontsize=\small,
    frame=lines,
    framesep=2mm]{java}
    // Client
    public void run() {
        try (Socket client = socket) {
            BufferedReader reader = new BufferedReader(new InputStreamReader(client.getInputStream()));
            BufferedWriter writer = new BufferedWriter(new OutputStreamWriter(client.getOutputStream()));

            for (int i = 0; i < cmd.length; i++) {
                writer.write(cmd[i]);
                writer.newLine();
            }
            writer.flush();

            for (int i = 0; i < cmd.length; i++) {
                String response = server.readLine();
                processResponse(response);
            }
        } catch (Exception e) {
            // ...
        }
    }
\end{minted}

Bei der TCP-Kommunikation sind schreibende Zugriffe nicht blocking, wohl aber lesende (s. Abschnitt~\ref{sec:sockettcp}).\\
Es macht deshalb keinen Unterschied, wenn die schreibenden Zugriffe in einer Schleife sind, und die lesenden ebenfalls.\\
Angenommen, ein schreibender Zugriff $w$ benötigt eine Zeit von $0$, ein lesender Zugriff $1$.\\
Bei $4$ schreibenden Zugriffen und $5$ lesenden Zugriffen addiert sich die benötigte Zeit zu $4*0 + 5*1 = 5$ - es ist im Endeffekt egal, in welcher Reihenfolge die Befehle ausgeführt werden - offenbar wächst die benötigte Zeit linear zu der Anzahl der lesenden Aufrufe.\\
Die Kommunikation zwischen Client und Server ist durch die Nutzung ein und desselben Sockets des Clients ohnehin \textit{sequentiell}.
Die Parallelität wirkt sich nur auf die Bearbeitung weiterer Clients aus.


\noindent
Wird nun statische und/ oder dynamische Parallelität (und = Mischform), so kann es durchaus sein, dass die Anfragen vom Client eher beantwortet werden, weil der Rechner mehr Resourcen hat, aber auch hier ist eine schnellere Abarbeitung nicht garantiert.
Bestünde eine Iteration der Kommunikation aus \textit{senden - empfangen - senden - empfangen ...} so kann es sein, dass die Kommunkationskette beim letzten \textit{empfangen} wegen der Last des Servers zum Halt kommt.
Gleiches kann aber auch durchaus passieren, wenn zuerst gesendet wird, und dann in einer Schleife empfangen wird.\\
Bis auf größere Kapazitäten des Servers, der dadurch in der Lage ist, mehrere Clients gleichzeitig zu bedienen, hat man dadurch also nicht viel gewonnen.

\subsection*{Wie kann die Datenstromorientierung des beschriebenen TCP-Clients zum Problem werden}
TCP besitzt die folgenden Eigenschaften:

\begin{itemize}
    \item es ist verlässlich durch Fluss- / Überlastkontrolle
    \item es ist datenstromorientiert
    \item es ist verbindungsorientiert
\end{itemize}

Gerade die Fluss-/ Überlastkontrolle könnte für den Client zum Problem werden, wenn er zuviele Nachrichten hintereinander sendet.\\
\noindent
Als Lösung könnte man ggf. versuchen, kürzere Nachrichten in einer größeren Nachricht zu sammeln, und diese an den Server zu schicken.\\
\noindent
Als Alternative böte sich ggfl. an, dass man eine Art ``Lastkontrolle`` auf dem Client implementiert, und immer nur $n$-Nachrichten hintereinander schickt.

\begin{tcolorbox}[colback=red!20,color=white,title=Anmerkung]
    Die Auswirkungen von Fluss-/Überlastkontrolle wurden im WS2023/2024 nicht ausführlich behandelt.\\
    Auch im Lehrbuch bzw. im Skript finden sich nicht viele Informationen hierzu.
\end{tcolorbox}

\subsection*{Wie kann eine Laufzeitverbesserung erreicht werden}
Man könnte ggfl. zwei verschiedene Thread-Klassen implementieren, die beide eine separate Socket-Verbindung zu dem Server aufbauen.\\
Dabei sendet der eine Client-Thread nur Daten, und der andere Client-Thread empfängt die Daten.\\
Hierzu müßte sich der empfangende Client-Thread, der sendet, zunächst identifzieren.
Der empfangende Client-Thread identifiziert sich mit denselben Daten.
Dadurch sollte der Server in der Lage sein, die zu sendenden Nachrichten auch den entsprechenden ``empfangenden`` Verbindungen zuordnen zu können..\\


\section{Aufgabe 4}
\subsection{Lösungsvorschlag}

Bei dem gegebenen Sourcecode fehlt in der Klasse \code{HelloImpl} der Konstruktor, der, wenn von \code{UnicastRemoteObject}\footnote{
s. ``Class UnicastRemoteObject``: \url{https://docs.oracle.com/en/java/javase/21/docs/api/java.rmi/java/rmi/server/UnicastRemoteObject.html} - abgerufen 16.2.2024
} abgeleitet wird, zwingend vorhanden sein muss (s. Abschnitt~\ref{sec:rmiapps})\footnote{
außerdem ist in der Aufgabenstellung ein Typo in ``implements``.
}.\\
Das ist dadurch bedingt, weil es in \code{UnicastRemoteObject} einen parameterlose Konstruktor gibt, der eine \code{throws}-Klausel besitzt, die eine Ausnahme vom Typ \code{RemoteException} wirft.
Da der in \code{HelloImpl} vom Compiler zur Verfügung gestellte Standardkonstruktor diese \code{throws}-Klausel nicht definiert, aber dennoch den parameterlosen Standardkonstruktor automatisch aufruft (wenn kein anderer Aufruf zu einem Konstruktor der Oberklasse erfolgt), kommt es zu einem Compiler-Fehler (vgl.\cite[313]{Oec22}).


\noindent
Die Klasse \code{String} implementiert das Interface \code{Serializable} und ist damit \texit{serialisierbar}.
Da der Rückgabewert als auch der formale Parameter der Methode \code{hello()} vom Typ \code{String} ist, und für Referenzübergabe ein \code{Remote}-Objekt exportiert werden muss, handelt es sich um \textbf{Wertübergabe}.\\
Für Referenzübergabe müsste der aktuelle Parameter für \code{hello()} exportiert sein - entweder, indem das Objekt explizit exportiert wird\footnote{
bspw. über ``exportObject`` - s. \url{https://docs.oracle.com/en/java/javase/21/docs/api/java.rmi/java/rmi/server/UnicastRemoteObject.html#exportObject(java.rmi.Remote,int)} - abgerufen 16.2.2024
}, oder indem ein Objekt einer Klasse verwendet wird, die von \code{UnicastRemoteObject} ableitet: Der Aufruf des Konstruktors von \code{UnicastRemoteObject} sorgt dafür, dass das Objekt implizit exportiert wird.\\
Die Klasse \code{String} implementiert nicht das \code{Remote}-Interface, kann also nicht exportiert werden\footnote{
Hierfür könnte dann aber ein Wrapper-Objekt genutzt werden, das exportiert werden kann und einen Zugriff auf einen String ermöglicht.}.\\

\noindent
Zusammenfassend gilt:

\blockquote[{\cite[344]{Oec22}}]{
    Die Vorgehensweise für Referenzübergabe lautet also: In der Schnittstelle muss als Typ eine RMI-Schnittstelle angegeben werden.
}

\noindent
Danach kann über einen Export dieses Objektes eine Referenzübergabe realisiert werden.\\

\noindent
Für weitere Informationen siehe Abschnitte \ref{sec:valuermi} und \ref{sec:refrmi}.\\

\noindent
Eine Implementierung des geg. Codes könnte wie folgt aussehen:

\begin{minted}[mathescape,
    linenos,
    numbersep=5pt,
    gobble=2,
    fontsize=\small,
    frame=lines,
    framesep=2mm]{java}
    public interface Hello extends Remote  {
        String hello(String name) throws RemoteException;
    }

    public class HelloImpl extends UnicastRemoteObject implements Hello{

        public HelloImpl() throws RemoteException {
        }

        public String hello(String name) throws RemoteException {
            return "hello" + name;
        }

    }

    public class HelloDemo {
        public static void main(String[] args) {
            Registry registry;
            if (args.length == 0) {
                try {
                    registry = LocateRegistry.createRegistry(8888);
                    registry.rebind("hello", new HelloImpl());
                } catch (RemoteException e) {
                    System.err.println("[server] " + e);
                }
            } else {

                try {
                    registry = LocateRegistry.getRegistry(8888);
                    Hello hello = (Hello)registry.lookup("hello");
                    System.err.println(hello.hello("World"));
                } catch (RemoteException | NotBoundException e) {
                    System.err.println("[client] " + e);
                }
            }
        }
    }
\end{minted}