\section{Verklemmungen (Deadlocks)}

 \subsection{Bedingungen für das Vorliegen einer Verklemmung}
Verklemmung entsteht, wenn es einen Zyklus von Threads gibt, so dass ein Thread auf ein Betriebsmittel wartet, welches der im Zyklus nachfolgende Thread besitzt.\\
Betriebsmittel können z.B. passive Objekte sein.\\
Voraussetzung ist, dass Betriebsmittel nur unter gegenseitigem Ausschluss benutzbar sind, einem Thread nicht entzogen werden können und dass Threads bereits Betriebsmittel besitzen und weitere anfordern.


Vermeidung:
\begin{itemize}
    \item Anforderung ``auf einen Schlag``
    \item Anforderung in festgelegter Reihenfolge
    \item[] (Bsp.: Alle Philosophen bis auf einen fordern zuerst linke, dann rechte Gabel an.)
    \item Weitere Verfahren: Wound-Wait- und Wait-Die-Verfahren, Bankier-Algorithmus
\end{itemize}
