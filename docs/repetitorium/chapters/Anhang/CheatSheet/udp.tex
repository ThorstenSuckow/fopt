\section{UDP}

\textbf{UDP} (\textit{User Datagram Protocol}) ist

\begin{itemize}
    \item verbindungslos
    \item nachrichtenorientiert
    \item unzuverlässig
\end{itemize}


\section*{Notizen}

\begin{itemize}
    \item DatagramPacket \textbf{receiving}: \code{DatagramPacket(byte[], int length, InetAddress address, int port)};
    \item DatagramPacket \textbf{sending}: \code{DatagramPacket(byte[], int length)};
    \item \code{ObjectInputStream.readObject()} wirft eine \textbf{checked exception} vom Typ \code{ClassNotFoundException}
\end{itemize}

\subsection*{Objekt senden}
\begin{itemize}
    \item \code{ByteArrayOutputStream} \code{baos} erzeugen (Länge p)
    \item \code{ObjectOutputStream} \code{oos} erzeugen (\code{baos} übergeben)
    \item \code{oos.writeObject()} mit Argument aufrufen, dann \code{oos.flush()}
    \item \code{baos.toByteArray()} aufrufen, in byte-Array \code{msg} speichern
    \item Streams schliessen
    \item DatagramPacket erzeugen und \code{msg} als Inhalt/Längenvorgabe übergeben, außerdem Empfänger-Adresse und Port
    \item über DatagramSocket versenden
\end{itemize}


\begin{minted}[mathescape,
    fontsize=\small]{java}
    String message = "Hello world!"
    ByteArrayOutputStream baos = new ByteArrayOutputStream(128);
    ObjectOutputStream oos = new ObjectOutputStream(baos);
    oos.writeObject(message);
    oos.flush();
    oos.close();
    baos.close();
    byte[] msg = baos.toByteArray();
    DatagramPacket p = new DatagramPacket(msg, msg.length);
    // senden...
\end{minted}

\subsection*{Objekt empfangen}
\begin{itemize}
    \item \code{DatagramPacket} \code{p} erzeugen (Argumente \code{new byte[n] b}, \code{n} übergeben, mit n = Länge)
    \item mit \code{DatagramSocket}s \code{receive(p)} empfangen
    \item \code{ByteArrayInputStream} \code{bais} erzeugen (\code{p.getData()} übergeben)
    \item \code{ObjectInputStream} \code{ois} erzeugen (\code{bais} übergeben)
    \item \code{ois.readObject()} aufrufen, \code{ClassNotFoundException} berücksichtigen
    \item streams schliessen
\end{itemize}


\begin{minted}[mathescape,
     fontsize=\small]{java}
    DatagramPacket p = new DatagramPacket(new byte[128], 128);
    // empfangen...
    ByteArrayInputStream bais = new ByteArrayInputStream(p.getData());
    ObjectInputStream ois = new ObjectInputStream(bais);
    Object o = ois.readObject();
    ois.close();
    bais.close();
\end{minted}

\subsection{Multicast}

\subsection*{Multicast-Kommunikation}

Mittels Multicast können sich mehrere Server auf eine Multicast-Adresse aufschalten.
Mehrere Rechner sind somit in der Lage, die Nachrichten einzelner Clients zu empfangen.
Damit die Nachrichten eines Clients an mehrere Server geht, muss der Client an die entsprechende Multicast-Adresse senden.
Server schalten sich mit Hilfe eines \code{MulticastSocket}s auf die Multicast-IP-Adresse auf.\\

\noindent
Multicast funktioniert i.d.R. nur in einem lokalen Netz.\\

\noindent
UDP mit Multicast kann bspw. dazu verwendet werden, um in einem Netz einen Server auszumachen, mit dem dann weiter über TCP-kommuniziert wird.\\
Hierzu sendet ein Client eine Anfrage an die Multicast-Adresse, bekommt von den aufgeschalteten Servern eine Antwort und kennt dadurch die IP-Adressen der Rechner (zumindest einen Teil, da UDP unzuverlässig).\\

\noindent
Das folgende Beispiel zeigt das Gerüst für die Erzeugung eines Multicast-Sockets\footnote{
es werden der Einfachheit halber die Methoden \textit{joinGroup()} / \textit{leaveGroup()} in ihrer \textit{deprecated} Form verwendet
}:

\begin{minted}[mathescape,
    fontsize=\small]{java}
    try (MulticastSocket msck = new MulticastSocket(port)) {
        msck.joinGroup(mcastaddr);

        while (true) {
            DatagramPacket p = new DatagramPacket(new byte[128], 128);
            msck.receive();
            ...
        }

        msck.leaveGroup(mcastaddr):
    } catch (IOException e) {
        ...
    }
\end{minted}