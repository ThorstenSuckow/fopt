\section{TCP}

\textbf{TCP} (\textit{Transmission Control Protocol}) ist

\begin{itemize}
    \item verbindungsorientiert
    \item datenstromorientiert
    \item zuverlässig mit Fluss-/Überlastkontrolle
\end{itemize}


\section*{Notizen}

\subsection*{Empfangen}
\begin{itemize}
    \item \textbf{try-with-resources} vor der while-schleife, nicht nach einem Sende-/Empfangsvorgang die Streams geschlossen werden
\end{itemize}


\begin{minted}[mathescape,
    fontsize=\small]{java}
    try (...) {
        while (true) {
            ... // empfangen, senden
                // exception beendet die Schleife, Verbindung zum
                // Client/Server wird geschlossen
        }
    } catch (Exception e) {
        ... // Fehlerbehandlung
    }
\end{minted}

\subsection*{Objekt empfangen}
\begin{itemize}
    \item \code{DatagramPacket} \code{p} erzeugen (Argumente \code{new byte[n] b}, \code{n} übergeben, mit n = Länge)
    \item mit \code{DatagramSocket}s \code{receive(p)} empfangen
    \item \code{ByteArrayInputStream} \code{bais} erzeugen (\code{p.getData()} übergeben)
    \item \code{ObjectInputStream} \code{ois} erzeugen (\code{bais} übergeben)
    \item \code{ois.readObject()} aufrufen, \code{ClassNotFoundException} berücksichtigen
    \item streams schliessen
\end{itemize}


\begin{minted}[mathescape,
     fontsize=\small]{java}
    DatagramPacket p = new DatagramPacket(new byte[128], 128);
    // empfangen...
    ByteArrayInputStream bais = new ByteArrayInputStream(p.getData());
    ObjectInputStream ois = new ObjectInputStream(bais);
    Object o = ois.readObject();
    ois.close();
    bais.close();
\end{minted}