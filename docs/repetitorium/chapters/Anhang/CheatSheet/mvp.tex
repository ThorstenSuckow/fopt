\section{MVP}


\subsection{Unterschied Architektur- / Entwurfsmuster}

\textbf{Entwurfsmuster} werden i.d.R. zur Lösung kleinere Teilbereiche von Software eingesetzt, indem sie das Zusammenwirken von Klassen und Schnittstellen beschreiben.\\

\noindent
\textbf{Architekturmuster} werden für die gesamte Software oder größere Teile einer Software eingesetzt und sind wie ein Bauplan für ganze Programmteile bzw. Teilbereiche einer Software zu verstehen.\\

\noindent
Architekturmuster helfen dabei, Anwendungen zu strukturieren, systematisch zu testen und fördern das Verständnis der Implementierung durch Strukturierung durch bestimmte Muster.

\subsection{MVP}

\subsection*{Model}
Das \textbf{Model} ist verantwortlich für

\begin{itemize}
    \item die Geschäftslogik
    \item Bereitstellung und das Ändern relevanter Daten für die Anwendung
    \item die Wahrung von Konsistenzbedingungen
\end{itemize}

\noindent
Das Model kennt bei MVP weder \textbf{Presenter} noch \textbf{View}

\subsection*{View}
Die \textbf{View} ist die Darstellungskomponente der Software und baut und verändert sie. \\
Benutzerinteraktionen werden an den Presenter weitergeleitet.\\
Mehrere Views können Teile des Models repräsentieren.\\
Die View kennt bei MVP nicht das \textbf{Model}.


\subsection*{Presenter}
Der \textbf{Presenter} vermittelt zwischen Model und View.
Er wird auch \textit{Ablaufsteuerungskomponente} oder \textit{Präsentationskomponente} genannt und weist die View zur Anzeige und zum Ändern von Daten an, die er aus dem Model bezieht.\\
Er nimmt in dem Muster eine \textbf{Vermittlerrolle} zwischen Model und View ein.\\
Er kann \textbf{Validierungslogik} beinhalten und implementiert die Ablaufsteuerung für vom Anwender ausgeführte Aktionen.\\



