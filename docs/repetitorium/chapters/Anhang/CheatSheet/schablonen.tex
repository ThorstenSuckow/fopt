\section{Methodenschablonen}

\subsection{Leser}

Der Leser verwendet kein \code{notify()} / \code{notifyAll()}, da er den Objektzustand nicht ändert.

\begin{itemize}
    \item mit \code{wait()} - Leser muss darauf warten, dass der Objektzustand gewisse Bedingungen erfüllt
    \item ohne \code{wait()} - Leser ist von keinem Objektzustand abhängig.
\end{itemize}


\subsection{Schreiber}

\begin{itemize}
    \item \textbf{ohne} \code{notify()} oder \code{notifyAll()}
    \begin{itemize}
        \item ohne \code{wait()} - Objektzustand muss keine Bedingung erfüllen, damit Daten geändert werden können.
        \item mit \code{wait()} - Schreiber muss auf einen best. Objektzustand warten, damit das Objekt geändert werden kann.
        \item[] $\rightarrow$ es müssen keine wartenden Threads darüber benachrichtigt werden, dass sich der Objektzustand geändert hat.
    \end{itemize}
    \item \textbf{mit} \code{notify()} oder \code{notifyAll()}
    \begin{itemize}
        \item ohne \code{wait()} - Objektzustand muss keine Bedingung erfüllen, damit Daten geändert werden können.
        \item mit \code{wait()} - Schreiber muss auf einen best. Objektzustand warten, damit das Objekt geändert werden kann.
        \item[] $\rightarrow$ in beiden Fällen wird \code{notify()} oder \code{notifyAll()} aufgerufen, damit evtl. wartende Threads ihre Wartebedingungen zum lesen/schreiben des Objektes erneut überprüfen können.
    \end{itemize}

\end{itemize}



