\section{Sessions, Cookies und ServletContext}

\subsection{Cookies}

\begin{itemize}
    \item werden aus dem \code{HttpServletRequest}-ausgelesen (\code{getCookies()}, HTTP-Protokoll: \textit{Cookie}-Header) und über das \code{HttpServletResponse}-Objekt an den Client gesendet (Hinzufügen eines Cookies: \code{addCookie()}, HTTP-Protokoll: \textit{Set-Cookie})
    \item existierende Cookies des Clients sind im \code{Cookie}-Header
    \item i.d.R. sendet der Client einmal gesetzte Cookies mit jeder Anfrage mit, es sei denn, sie sind zwischenzeitlich gelöscht (manuell / automatisch durch einstellbar lange Zeit)
\end{itemize}


\subsection{Sessions}
\begin{itemize}
    \item von einem \code{HttpServletRequest} bekommt man immer dasselbe \code{HttpSession}-Objekt, wenn die Http-Anfragen von demselben Benutzer stammt.
    \item es lassen sich beliebige anwendungsabhängige Daten in einem Session-Objekt ablegen.
    \item klassisches Einsatzgebiet: Warenkörbe in Online-Shops.
    \item Zugriff auf Session muss synchronisiert sein, da ein Nutzer mehrere HTTP-Anfragen für ein und dieselbe Session senden kann.
    \item Sessions und die damit verbundenen Daten werden automatisch gelöscht, wenn die Session eine (einstellbar lange) Zeit nicht mehr aktiv gewesen ist.
    \item eine Session wird i.d.R. über ein \textit{Session-Cookie} realisiert, in dem ein \textit{Wert} gespeichert ist (Schlüssel: \textit{JSESSIONID}), mit dem das Sitzungsobjekt auf dem Server assoziiert ist
\end{itemize}


\subsection{ServletContext}
\begin{itemize}
    \item Anwendungsglobale Daten lassen sich in einem \code{ServletContext}-Objekt ablegen
    \item dieses Objekt enthält eine Attributliste, die man mit \code{getAttribute()} / {\code{setAttribute()} / \code{removeAttribute()} auslesen / anpassen kann.
    \item es können Objekte beliebigen Typs gespeichert werden.
    \item wenn von Servlets Daten in einem \textit{ServletContext}-Objekt gespeichert werden sollen, kann über die Annotation \code{WebServlet} und deren Parameter \code{loadOnStartup} eine gewünschte Initialisierungsreihenfolge angeben (Standard = $-1$: wird nicht automatisch initialisiert, sondern erst bei Bedarf).
    \item die Methode \code{init()} eines Servlets wird aufgerufen, wenn das Servlet initialisiert wurde.
    Hier kann über \code{getServletContext()} auf das \textit{ServletContext}-Objekt zugegriffen werden.
\end{itemize}

