\section{Statisch und Dynamisch Parallele TCP / UDP Server}

\subsection{Statisch Parallel}

Bei statischer Parallelität werden Felder von Threads konstanter Größe verwendet.
\subsubsection{TCP}

\begin{itemize}
    \item jedes Feld enthält einen Thread, der dasselbe \code{ServerSocket}-Objekt übergeben bekommt
    \item die \code{run()}-Methode wartet mit \code{accept()} auf Verbindungseingang und bedient die Verbindungen
\end{itemize}


\subsubsection{UDP}

\begin{itemize}
    \item jedes Feld enthält einen Thread, der dasselbe \code{DatagramSocket}-Objekt übergeben bekommt
    \item die \code{run()}-Methode wartet mit \code{receive(p: DatagramPacket)} auf Nachrichten und bearbeitet diese
\end{itemize}

\subsection{Dynamisch Parallel}


\subsubsection{TCP}

\begin{itemize}
    \item für jeden Verbindungseingang mit \code{accept()} wird ein neuer Thread erstellt, der die Verbindung in der \code{run()}-Methode
    bedient
\end{itemize}


\subsubsection{UDP}

\begin{itemize}
    \item für jeden Nachricht wird zur Bearbeitung ein neuer Thread erstellt, der das originäre \code{DatagramSocket}-Objekt übergeben bekommt,
    über das Antworten an die Clients gesendet werden können.
\end{itemize}
