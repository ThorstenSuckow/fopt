\section{JavaFX}


\subsection{Kochrezept für JavaFX-Anwendungen}

\begin{itemize}
    \item Klasse erstellen, die von \code{Application} ableitet
    \item in der \code{main(String... args)}-Method \code{launch(args)} aufrufen
    \item \code{start(Stage primaryStage)} überschreiben:
    \begin{itemize}
        \item Interaktionselemente und Containerelemente definieren (``Aufbau der Baumstruktur``)
        \item Ereignisbehandlung implementieren
        \item root-Element der im vorherigen Schritt definierten Elemente der \code{Scene} übergeben, die über \code{primaryStage.setScene()} in die Stage eingefügt wird
        \item \code{primaryStage} mit Titel und ggf. Dimensionen konfigurieren, dann \code{show()} auf der Stage aufrufen.
    \end{itemize}
\end{itemize}

\subsection*{Notizen }

\begin{itemize}
    \item Shapes, die dynamisch erstellt werden (bspw. durch Mausereignisse), auch dem betreffenden Container
    (bspw. Pane) hinzufügen
    \item \code{VBox.setVgrow(container, Priority.ALWAYS)} sorgt dafür, dass ein Container an die vertikale Größe des Elterncontainers angepaßt wird
    \item Rückgabewert von \code{CheckBox.}\code{selectedProperty()} ist vom Typ \code{Property<Boolean>}, die formalen Parameter für einen \textit{ChangeListener} lauten dann folglich \code{ObservableValue<? extends Boolean> obs}, \code{boolean oldValue}, \code{boolean newValue}
\end{itemize}