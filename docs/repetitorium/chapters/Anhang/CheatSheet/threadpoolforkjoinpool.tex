\section{Thread-Pool und ForkJoinPool}

\subsection{Thread-Pool}

\begin{itemize}
    \item es muss nicht für jeden Auftrag eine neuer Thread erstellt werden.
    \item Anzahl der Threads ist nach oben beschränkt (nicht zwingend konstant, kann schwanken).
    \item Threads teilen sich die Aufträge - ein Thread kann einen anderen Auftrag bearbeiten, sobald er fertig ist.
\end{itemize}

\subsection{ForkJoinPool}

\begin{itemize}
    \item realisiert einen Thread-Pool, der speziell für baumartige Berechnungen gedacht ist
    \item alle Threads des ForkJoinPools sind Hintergrund-Threads.
    \item ein Threads eines ForkJoinPools kann weitere Aufträge bearbeiten, sobald er mit \code{join} auf das Ende eines anderen Auftrages wartet: Ein Thread des Pools, der einen neuen Auftrag erteilt und auf dessen Ende wartet, ist während des Wartens nicht für andere Aufträge blockiert.
    \item[] $\rightarrow$ Threads in einem ForkJoinPool können Aufträge in den Pool werfen, die sie zur Berechnung benötigen, und während des Wartens andere Aufträge ausführen
\end{itemize}
