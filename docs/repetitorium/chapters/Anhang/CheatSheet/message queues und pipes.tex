\section{Message Queues und Pipes}

\textbf{Message Queues} $\righarrow$ Nachrichtenorientiert (UDP)\\

\noindent
\textbf{Pipes} $\righarrow$ Datenstromorientiert (TCP)\\

\noindent
Das Senden bei Pipes erfolgt i.d.R. als unteilbare Aktion.
\begin{itemize}
    \item Wenn die Nachricht größer ist als der \textit{im Puffer verbleibende Platz}, wird gewartet, bis der Platz frei ist - dann wird \textbf{auf einen Schlag} in den Puffer kopiert
    \item Ist die Nachricht länger als \textit{die Größe des Puffers}, wird die Nachricht geteilt und in Portionen gesendet - dadurch kann es vorkommen, dass Nachrichtenteile durchgemischt werden; sind die Nachrichtenteile nicht größer als die Pufferlänge, befinden sie sich komplett im Puffer;  umgekehrt ist es möglich, dass Stücke von Nachrichtenteilen ``durchgemischt`` werden (vgl.~\cite[117 f.]{Oec22}).
\end{itemize}

Beim Empfangen wird nur so lange gewartet, bis der Puffer nicht mehr leer ist:
\begin{itemize}
    \item der Empfänger gibt an, wie viele Bytes ($n$) er lesen möchte.
    \item ist der Puffer komplett leer, wird mit dem Lesen gewartet, bis Daten im Puffer sind.
    \item Es werden $n$ Daten aus dem Puffer gelesen - sind weniger als $n$ Daten im Puffer, werden auch nur soviele Daten gelesen (die Daten werden hierbei in ein Feld passender Länge kopiert).
    \item $\rightarrow$ der Empfänger beginnt nicht erst mit dem Lesen, wenn $n$ Daten vorhanden sind
\end{itemize}
