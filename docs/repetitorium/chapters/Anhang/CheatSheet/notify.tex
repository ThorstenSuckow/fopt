\section{notify() / notifyAll()}

\subsection*{notify() verwendet man...}
\begin{itemize}
    \item in der Warteschleife befinden sich ausschließlich Threads mit einer identischen Wartebedingung
    \item[] \textbf{UND}
    \item wenn es nur einen Thread gibt, der seine \textit{while-wait-Schleife} verlassen kann bzw. soll.
\end{itemize}\\

\noindent
Beispiele:\\

\noindent
\code{SynchStack}: Es gibt eine Wartebedingung (\code{isEmpty()}) und nur \textit{einen} Thread, der davon profitiert, aus der Warteschleife entlassen zu werden: Sind mehrere Threads $n$ in der Warteschlange,  haben $n$ Aufrufe von \code{pop()} ergeben, dass der Stack leer gewesen ist.\\
War der Stack leer, und es erfolgt ein Aufruf von \code{notify()} \textit{nach} einem \code{push()}, wird \textit{einer} dieser Threads aus der Warteschlange geholt, damit er seine Operation durchführen kann - danach wird der Stack wieder leer sein, also würden nicht mehrere Threads davon profitieren, aus der Warteschlange geholt zu werden.\\
(Sind genug Elemente im Stack, und es wird \code{notify()} aufgerufen, ist der Aufruf ohnehin ohne Effekt, da sich keine Threads in der Warteschlange befinden).

\subsection*{notifyAll() verwendet man...}
\begin{itemize}
    \item wenn sich mehrere Threads mit unterschiedlichen Wartebedingungen in der Warteschlange befinden.
    \item[] \textbf{ODER}
    \item durch die Änderung des Objektzustandes können mehrere Threads ihre \textit{while-wait-Schleife} verlassen \footnote{
     sie profitieren damit, aus der Warteschlange herausgelassen zu werden.
    }
\end{itemize}\\

\noindent
Beispiele:\\

\noindent
\code{BoundedCounter}: Ein Thread, der davon profitiert, aus der Warteschlange herausgelassen zu werden; mehrere Wartebedingungen.\\
Ähnlich dem SynchStack, aber jetzt gibt es mehrere Wartebedingungen.\\
Wird \code{up()} aufgerufen, kann es sein, dass die Threads, die \code{down()} aufrufen wollen, in der Warteschlange sind.
Haben $n$ Aufrufe \code{value==min} ergeben, und wurde zwischenzeitlich die Warteschlange nicht verändert, befinden sich auch $n$ Threads in der Warteschlange.\\
Nur einer profitiert nun davon, den Wert wieder herunterzählen zu können, weil \code{value} nach einem \code{up()} \code{min+1} ist.
\textbf{Oder} es existiert andersherum ein Thread, der vom Hochzählen profitiert.\\
Es befinden sich aber niemals Threads in der Warteschlange, die auf ein Hochzählen \textit{und} herunterzählen warten.\\

\noindent
\code{ital. Ampel}: Mehrere Threads, die davon profitieren, aus der Warteschlange herausgelassen zu werden (``nebeneinander fahrende Autos``, wenn die Ampel grün zeigt, können diese ihre \textit{while-wait-Schleife} verlassen), eine Wartebedingungen (Ampel: rot)\\

\noindent
\code{EventSet}: Mehrere Threads, die davon profitieren, aus der Warteschlange herausgelassen zu werden (\textit{waitAND}, \textit{waitOR}), mehrere Wartebedingungen (alle \code{true}, mind. einer \code{true}).







