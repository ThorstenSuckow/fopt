\section{Interrupt-Flag}

Mit dem Interrupt Flag ist es möglich, die ``Blockade``\footnote{
bzgl. der Begrifflichkeit ``Blockade`` siehe Abschnitt~\ref{subsec:threadend}
} eines Threads zu unterbrechen, bspw. weil er sich gerade in \code{join()}, \code{wait()} oder \code{sleep()} befindet.

\begin{itemize}
    \item \code{isInterrupted()} setzt nicht den \textit{interrupted} status des Threads zurück.
\end{itemize}

\begin{minted}[mathescape,
    linenos,
    numbersep=5pt,
    gobble=2,
    fontsize=\small]{java}
    public void run() {

        try {
            while (!isInterrupted() {
                Thread.sleep(1000);
            }

        } catch (InterruptedException e) {
            ...
        }


    }
\end{minted}