\section{Locks and Conditions}

\subsection{Locks}

\subsubsection*{Vorteile}

\begin{itemize}
    \item Teil der concurrent-Klassenbibliothek in Java.
    \item \code{tryLock()} ermöglicht es, eine Sperre zu setzen, falls das Objekt noch nicht gesperrt wurde (Rückgabewert \code{true}, ansonsten \code{false}.
    Auch möglich unter Angabe eines timeouts.\footnote{
s. ``tryLock``: \url{https://docs.oracle.com/javase/8/docs/api/java/util/concurrent/locks/Lock.html#tryLock--} - abgerufen 02.03.2024
    }).
    \item ermöglicht das Setzen \code{lock()} / Entfernen \code{unlock()} von Sperren im Code auch über Methoden hinweg\footnote{
        grob formuliert wirkt sich \textit{synchronized} auf eine Methode aus
    }, aber nur \textit{von ein und demselben} Thread\footnote{Lehrbuch: das kann als Vor-/Nachteil gesehen werden - \cite[149]{Oec22}}.
    \item man kann auf das Setzen einer Sperre befristet warten.
    \item man kann das Warten eines Threads auf das Setzen einer Sperre durch \code{interrupt()} unterbrechen.
    \item man kann Locks für unterschiedliche Kontexte erstellen, bspw. Lese- und Schreibsperren.
    \item man kann sich Implementierungen vorstellen, die vor dem Setzen einer Sperre überprüfen, ob eine Verklemmung eintreten kann, und das Setzen der Sperre abweisen
\end{itemize}


\subsubsection*{Nachteile}
\begin{itemize}
    \item Kein \code{lock()} ohne \code{unlock()} - es muss daran gedacht werden, die Sperre wieder zu entfernen (am besten in einem \code{finally}-Block)
\end{itemize}

\subsection{Conditions}

\subsubsection*{Vorteile}

\begin{itemize}
    \item statt \textit{einer oder alle} (\code{notify()} / \code{notifyAll()}) können Threads über (beliebig) viele Conditions gezielt nach Wartebedingung geweckt werden.
\end{itemize}
