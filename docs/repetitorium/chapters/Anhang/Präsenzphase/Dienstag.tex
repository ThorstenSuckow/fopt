\section{Dienstag}

\subsection*{ToggleGroups}
\code{RadioButton}s lassen sich über eine \code{ToggleGroup}\footnote{
``Class ToggleGroup``: \url{https://docs.oracle.com/javase/8/javafx/api/javafx/scene/control/ToggleGroup.html} - abgerufen 20.2.2024
} gruppieren:

\begin{minted}[mathescape,
    linenos,
    numbersep=5pt,
    gobble=2,
    frame=lines,
    framesep=2mm]{java}
    RadioButton one = new RadioButton("one");
    RadioButton two = new RadioButton("two");
    RadioButton three = new RadioButton("three");

    ToggleGroup radioButtonGroup = new ToggleGroup();
    radioButtonGroup.getToggles().addAll(one, two, three);
\end{minted}\\

\subsection*{StringProperty an IntegerProperty binden}

\begin{center}\code{Bindings.convert(observableValue: ObservableValue<?>): StringExpression}\end{center}
ermöglicht es, bsp. ein JavaFX-\code{Label} an eine \code{SimpleIntegerProperty} zu binden\footnote{
``convert``: \url{https://docs.oracle.com/javase/8/javafx/api/javafx/beans/binding/Bindings.html#convert-javafx.beans.value.ObservableValue-} - abgerufen 20.2.2024
}.\\

\noindent
Das Folgende Beispiel bindet 3 Labels an drei verschiedene \code{SimpleIntegerProperty}s.\\
Die Summe dieser Properties wird in dem Label code{totalLabel} angezeigt:
\begin{minted}[mathescape,
    linenos,
    numbersep=5pt,
    gobble=2,
    frame=lines,
    framesep=2mm]{java}
    Label lineCountLabel = new Label();
    Label rectangleCountLabel = new Label();
    Label circleCountLabel = new Label();
    Label totalLabel = new Label();

    lineCountLabel.textProperty().bind(Bindings.convert(lineCountProperty));
    rectangleCountLabel.textProperty().bind(Bindings.convert(rectangleCountProperty));
    circleCountLabel.textProperty().bind(Bindings.convert(circleCountProperty));

    totalLabel.textProperty().bind(
        Bindings.convert(lineCountProperty.add(rectangleCountProperty).add(circleCountProperty))
    );
\end{minted}\\

\subsection*{Stage Owner}

Mit \code{initOwner(owner: Window)}\footnote{
``initOwner``: \url{https://docs.oracle.com/javase/8/javafx/api/javafx/stage/Stage.html#initOwner-javafx.stage.Window-} - abgerufen 20.2.2024
} läßt sich der \textbf{Owner} eines Windows festlegen.
Das Fenster, auf das \code{initOwner()} aufgerufen wurde, wird dann bspw. geschlossen, wenn der Owner geschlossen wird:

\blockquote[{\url{https://docs.oracle.com/javase/8/javafx/api/javafx/stage/Stage.html} - abgerufen 20.2.2024}]{
    A stage can optionally have an owner Window. When a window is a stage's owner, it is said to be the parent of that stage. When a parent window is closed, all its descendant windows are closed. The same chained behavior applied for a parent window that is iconified. A stage will always be on top of its parent window. The owner must be initialized before the stage is made visible.
}

\subsection*{ObservableList: removeAll() vs. clear()}

Als Alternative zu der Methode \code{removeAll(c: Collection<?>)} des Interface \code{java.util.List<E>}\footnote{
``Interface List<E>``: \url{https://docs.oracle.com/en/java/javase/21/docs/api/java.base/java/util/List.html} - abgerufen 20.2.2024
} bietet die davon abgeleitete Schnittstelle \code{ObservableList<E>} aus dem Package \code{javafx.collections} die Methode
\code{removeAll(elements:E...)}\footnote{
    ``removeAll``\url{https://docs.oracle.com/javase/8/javafx/api/javafx/collections/ObservableList.html#removeAll-E...-} - abgerufen 20.2.2024
} an, die einen Aufruf mit \textbf{variabler Argumentanzahl}\footnote{
    ``8.4.1. Formal Parameters - VariableArityParameter``:  \url{https://docs.oracle.com/javase/specs/jls/se21/html/jls-8.html#jls-VariableArityParameter} - abgerufen 20.2.2024
} erlaubt.\\

\noindent
Hierbei ist zu beachten, dass ein Aufruf von \code{removeAll()} ohne Argumente \textit{nicht} automatisch alle Objekte aus der Liste löscht, wie man aufgrund der semantischen Bedeutung der Methode vermuten könnte.\\
Hierzu sollte stattdessen die Methode \code{clear()}\footnote{
``clear``: \url{https://docs.oracle.com/en/java/javase/21/docs/api/java.base/java/util/List.html#clear()} - abgerufen 20.2.2024
} verwendet werden.
