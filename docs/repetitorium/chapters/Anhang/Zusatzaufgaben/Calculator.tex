\section{Calculator}\label{ch:calculator}


\subsection{Lösung}

Die Lösung enthält nicht die fxml-Datei\footnote{
vollständige Lösung unter \url{https://github.com/ThorstenSuckow/fopt/tree/main/src/main/java/gui/calculator}
}.

\begin{minted}[mathescape,
    linenos,
    numbersep=5pt,
    gobble=2,
    fontsize=\small,
    frame=lines,
    framesep=2mm]{java}
    public class Calculator extends Application {
        private boolean clearOnInput = true;

        public void start(Stage primaryStage) throws IOException {
            Pane root;
            root = FXMLLoader.load(getClass().getResource("mainui.fxml"));

            installListeners(root);

            Scene scene = new Scene(root);
            primaryStage.setScene(scene);

            primaryStage.show();
        }

        private void installListeners(Parent parent) {
            Set<Node> buttons = parent.lookupAll(".button");

            for (Node button : buttons) {
                Button b = (Button) button; // casting is required since
                                            // Node does not provide a setOnAction-method
                b.setOnAction(this::onButtonAction);
            }
        }

        private void onButtonAction(ActionEvent evt) {
            Button b = (Button) evt.getSource();
            handleInput(b.getText(), (TextField) ((Button) evt.getSource()).getScene().getRoot().lookup("#display"));
        }

        private void handleInput(String btnText, TextField display) {

            if (clearOnInput) {
                display.setText("");
                clearOnInput = false;
            }

            String displayText = display.getText();

            switch (btnText) {
                case "CLEAR":
                    display.setText("");
                    return;
                case "DELETE":
                    display.setText(
                        displayText.length() > 0
                            ? displayText.substring(0, display.getText().length() - 1)
                            : ""
                    );
                    return;
                case "=":
                    try {
                        display.setText(displayText + "=" + Computation.evaluate(displayText));
                    } catch (IllegalArgumentException e) {
                        display.setText(displayText + "=>FEHLER");
                    }
                    clearOnInput = true;
                    return;
                default:
                    display.setText(display.getText() + btnText);
            }
        }

        public static void main(String[] args) {
            launch(args);
        }

    }
\end{minted}\\


\subsection{Anmerkung und Ergänzungen}


\begin{itemize}
    \item \code{Button}s können auch über \code{lookupAll()} ermittelt werden - als Wert übergibt man der Methode einfach \code{.button}, was den CSS-Selektor für die CSS-Klasse \textbf{button} ist, die per Default einem \code{Button}-Objekt zugewiesen ist.
    \item Die Methode \code{launch} ist \textbf{blocking}, aus ihr kehrt man erst zurück, wenn  alle Anwendungsfenster geschlossen wurden oder \code{Platform.exit()}\footnote{
    beendet den \textbf{JavaFX Application Thread} - s. \url{https://docs.oracle.com/javase/8/javafx/api/javafx/application/Platform.html#exit--} - abgerufen 8.2.2024
    } aufgerufen wurde:
    \blockquote[{``launch``: \url{https://docs.oracle.com/javase/8/javafx/api/javafx/application/Application.html#launch-java.lang.String...-} - abgerufen 8.2.2024}]{
     The launch method does not return until the application has exited, either via a call to Platform.exit or all of the application windows have been closed.
    }
\end{itemize}
