\chapter{ASB Zusatzaufgabe EventSet}\label{ch:eventset}


\section{Lösung}

\begin{minted}[mathescape,
    linenos,
    numbersep=5pt,
    gobble=2,
    fontsize=\small,
    frame=lines,
    framesep=2mm]{java}
    public class EventSet {
        final private boolean[] set;
        public EventSet(int n) {
            if (n < 0) {
                throw new IllegalArgumentException();
            }
            set = new boolean[n];
        }

        public synchronized void set(int pos, boolean value) {
            if (pos < 0 || pos >= set.length) {
                throw new IllegalArgumentException();
            }
            this.set[pos] = value;
            if (value) {
                notifyAll();
            }
        }

        public synchronized void waitAND() {
            while(!allTrue()) {
                try {
                    this.wait();
                } catch (InterruptedException ignored) {}
            }
            notifyAll();
        }

        public synchronized void waitOR() {
            while (!oneTrue()) {
                try {
                    this.wait();
                } catch (InterruptedException ignored) {}
            }
            notifyAll();
        }

        private boolean allTrue() {
            for (boolean b : set) {
                if (!b) {
                    return false;
                }
            }
            return true;
        }

        private boolean oneTrue() {
            for (boolean b : set) {
                if (b) {
                    return true;
                }
            }
            return false;
        }
    }
\end{minted}\\


\section{Anmerkung und Ergänzungen}

\begin{itemize}
    \item Laut Aufgabenstellung sollen nur negative Argumente für den Konstruktor ausgeschlossen werden.
    Es ist okay, das Feld mit einer Länge von $0$ zu initialisieren.
    \item \code{waitOR} und \code{waitAND} haben in der Aufgabenstellung keinen expliziten Rückgabetyp,
    also kann für die Methode der Rückgabetyp \code{void} deklariert werden.
    \item Es sind zwei unterschiedliche Wartebedingungen vorhanden, also muss wieder mit  \code{notifyAll()}
    gearbeitet werden, damit alle Threads ihre Wartebedingungen überprüfen können.
    \item \code{set()} muss ebenfalls über \code{notifyAll()} alle Threads zur erneuten Überprüfung ihrer Wartebedingungen wecken - aber nur, wenn das Feld in einem Index auf \code{true} gesetzt wurde:
        \begin{itemize}
            \item $T^{AND}_{1..n}$ können nur weiterlaufen, wenn alle Einträge im Feld auf \code{true} gesetzt sind.
            \item $T^{OR}_{1..n}$ können nur weiterlaufen, wenn mindestens ein Eintrag im Feld auf \code{true} gesetzt ist.
            Sind alle Einträge im Feld auf $0$, werden die Threads $T^{OR}_{1..n}$ in die Warteschlange des Objektes
            eingereiht.
            Wird nun ein Eintrag auf \code{true} gesetzt, werden diese Threads wieder aus der Warteschlange geholt, und
            die Threads konkurrieren erneut um die Sperre des \code{EventSet}-Objektes - und zwar so lange, bis wieder alle Einträge auf $0$ sind - die Wartebedingung greift dann wieder, und sie werden in die Warteschlange eingereiht.
        \end{itemize}
\end{itemize}
