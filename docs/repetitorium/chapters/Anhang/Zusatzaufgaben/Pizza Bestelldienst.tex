\section{Pizza Bestelldienst}\label{ch:pizzaservlet}


\subsection{Lösung}

Im Folgenden wird das \code{PizzaServlet} auszugsweise vorgestellt\footnote{
vollständige Implementierung unter \url{https://github.com/ThorstenSuckow/fopt/tree/main/ee/foptweb/src/main/java/servlets/pizza}
}.\\
Generell folgt die Lösung dem Ansatz von \textbf{Sessions}, dass nämlich in einem zentralen Container zu einer \textbf{Cookie-ID} Bestellungen
assoziiert sind.

\begin{minted}[mathescape,
    linenos,
    numbersep=5pt,
    gobble=2,
    fontsize=\small,
    frame=lines,
    framesep=2mm]{java}
    @WebServlet(urlPatterns="/zusatzaufgaben/pizza-servlet")
    public class PizzaServlet extends HttpServlet {

        private final static int MAX_AGE = 10;
        private HashMap<String, Bestellung> bestellungen;

        private int bestellungId;
        record Bestellung(String kundenname, String kundennummer, String adresse, String pizzaliste, long letzteBestellung){
            public String toString() {
                return "kundenname: " + kundenname +
                        "; kundennummer: " + kundennummer +
                        "; adresse: " + adresse +
                        "; pizzaliste: " + pizzaliste;
            }
        }

        public void init() {
            bestellungId = 0;
            bestellungen = new HashMap<>();
        }

        public synchronized void doGet(HttpServletRequest request, HttpServletResponse response) {
            response.setContentType("text/html");
            // writer von response anfordern und Formular übergeben
        }
        public synchronized void doPost(HttpServletRequest request, HttpServletResponse response) {
            response.setContentType("text/html");

            Cookie pizzaCookie = getPizzaCookie(request);
            if (pizzaCookie == null) {
                pizzaCookie = createPizzaCookie();
            }

            Bestellung bestellung = new Bestellung(
              // ...
            );

            bestellungen.put(pizzaCookie.getValue(), bestellung);
            response.addCookie(pizzaCookie);

            try {
                response.getWriter().print(
                        "Vielen Dank für ihre Bestellung! <br/>" +
                            bestellung
                );
            } catch (IOException e) {
                response.setStatus(500);
            }
        }

        private String getForm(HttpServletRequest request) {

            Cookie pizzaCookie = getPizzaCookie(request);
            Bestellung bestellung = getBestellungFromCookie(pizzaCookie);

            String kundenname = "";
            String kundennummer = "";
            String adresse = "";
            String pizzaliste = "";

            if (bestellung != null) {
                kundenname = bestellung.kundenname();
                kundennummer = bestellung.kundennummer();
                adresse = bestellung.adresse();
                pizzaliste = bestellung.pizzaliste();
            }

            // formular erstellen und zurückgeben
            // ...
        }

        private Bestellung getBestellungFromCookie(Cookie pizzaCookie) {
            if (pizzaCookie != null) {
                String key = pizzaCookie.getValue();
                return bestellungen.get(key);
            }
            return null;
        }

        private Cookie getPizzaCookie(HttpServletRequest request) {
            Cookie pizzaCookie = null;
            if (request.getCookies() != null) {
                for (Cookie cookie: request.getCookies()) {
                    if (cookie.getName().equals("pizzaCookie")) {
                        pizzaCookie = cookie;
                        break;
                    }
                }
            }

            if (pizzaCookie != null) {
                String key = pizzaCookie.getValue();
                Bestellung bestellung = bestellungen.get(key);
                if (bestellung == null) {
                    pizzaCookie = null;
                } else {
                    long letzteBestellung = bestellung.letzteBestellung();
                    if (System.currentTimeMillis() - letzteBestellung > MAX_AGE) {
                        pizzaCookie = null;
                        bestellungen.remove(key);
                    }
                }
            }

            return pizzaCookie;
        }

        private Cookie createPizzaCookie() {
            Cookie pizzaCookie = new Cookie("pizzaCookie", "bestellung" + (bestellungId++));
            pizzaCookie.setMaxAge(MAX_AGE);
            return pizzaCookie;
        }
}
\end{minted}\\


\subsection{Anmerkung und Ergänzungen}

Das Servlet wird mit \code{@WebServlet} annotiert, gefolgt von einem Klammernpaar (rund), in dem \code{Element-Wert-Paare}\footnote{
``element-value pairs``: \url{https://docs.oracle.com/javase/specs/jls/se21/html/jls-9.html#jls-ElementValuePairList} - abgerufen 13.2.2024
} angegeben werden können:

\begin{minted}[mathescape,
    numbersep=5pt,
    gobble=2,
    framesep=2mm]{java}
    @WebServlet(name="pizza-servlet",urlPatterns="/pizza-servlet")
\end{minted}
Der generelle Ablauf beim Aufbereiten einer Response ist:

\begin{itemize}
    \item Setzen des \textbf{Content-Type}-Header
    \item Anfragen des \code{Writer}s von dem Response-Objekt, Übergabe der zu sendenden Daten an den \code{PrintWriter} mittels \code{print(s: String)}, \textit{commit} der Ausgabedaten über \code{flush()}\footnote{
    s. ``getWriter``: \url{https://jakarta.ee/specifications/servlet/6.0/apidocs/jakarta.servlet/jakarta/servlet/servletresponse#getWriter()} - abgerufen 13.2.2024
    }
\end{itemize}


\begin{itemize}
    \item Da sowohl bei einem \textbf{POST} als auch einem \textbf{GET} auf zentrale Daten lesend und schrieben zugegriffen werden, sind die Methoden synchronisiert: Bei dem Objekt, das Ziel der Datenzugriffe ist, handelt es sich um die \code{HashMap}, die zu den Cookie-IDs (Schlüssel) die Bestellungen (vom Typ eines Bestellung-\code{Record}s) vorhält.
    \item In der Aufgabenstellung wird nach einer Cookie-basierten Lösung gefragt, die in der umgesetzten Form eigentlich einem Session-Management über \textit{Tokens} entspricht -
     der Cookie speichert eine Id (den \textit{Token}), die Daten hierzu werden auf dem Server gespeichert und sind mit diesem Token assoziiert.
    Der Entwickler muss sich sowohl um das Neuinitialisieren der Tokens kümmern als auch um das Löschen der server-seitig-gespeicherten Daten, sollte ein Cookie aufgrund der \textbf{max-age} Einstellung (10 sek. in der Aufgabe) invalidiert werden müssen.\\
    Bie Sessions wird i.d.R der Session-Cookie automatisch gelöscht, wenn die zugehörige Browserinstanz geschlossen wird (s. Abschnitt~\ref{sec:cookieaccess}).\\
    Auch ist bei Sessions die Semantik (``\textit{Inaktivität}``) für das Setzen eines Timeouts eine andere, dort über die Schnittstellenmethode \begin{center}\code{setMaxInactiveInterval(interval: int)}\end{center}\footnote{
        s. ``setMaxInactiveInterval`` : \url{https://jakarta.ee/specifications/servlet/6.0/apidocs/jakarta.servlet/jakarta/servlet/http/httpsession#setMaxInactiveInterval(int)} - abgerufen 13.2.2024
    }, außerdem kann auch der Zeitpunkt abgefragt werden, an dem die Session zuletzt über einen Request aktiv gewesen ist\footnote{``getLastAccessedTim``: \url{https://jakarta.ee/specifications/servlet/6.0/apidocs/jakarta.servlet/jakarta/servlet/http/httpsession#getLastAccessedTime()} - abgerufen 13.2.2024}.
    \item Enthält ein Request keine Cookies, liefert \code{getCookies():Cookie[]} den Wert \code{null} zurück\footnote{
    ``getCookies``: \url{https://jakarta.ee/specifications/servlet/6.0/apidocs/jakarta.servlet/jakarta/servlet/http/httpservletrequest#getCookies()} - abgerufen 13.2.2024
    }
    \item Bis auf den Cookie-Namen und den Wert des Cookies werden sonst keine Informationen von dem Client an den Server mit dem Cookie-Header gesendet\footnote{
    s. ``Cookie``: \url{https://developer.mozilla.org/en-US/docs/Web/HTTP/Headers/Cookie}
    }.\\
    Deshalb kann in der o.a. Implementierung auch nicht über den Request auf den vorher konfigurierten Wert für \textbf{max-age} zugegriffen werden - diese Information wird vom Client-Browser gespeichert, aber nicht an den ``origin server`` zurückkommuniziert\footnote{
    s. `´4.2.1.  Syntax``: \url{https://www.rfc-editor.org/rfc/rfc6265#section-4.2.1} bzw. im aktuelleren Draft \url{https://httpwg.org/http-extensions/draft-ietf-httpbis-rfc6265bis.html#name-syntax-2} - beide abgerufen 13.2.2024
    }
\end{itemize}