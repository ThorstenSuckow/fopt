\section{Sinusfunktion}\label{ch:sinusfunktion}


\subsection{Lösung}

Im Folgenden die Implementierung der Methode \code{drawSinusCurve()} aus der Klasse \code{SinusPresenter}\footnote{
    vollständige Lösung unter \url{https://github.com/ThorstenSuckow/fopt/tree/main/src/main/java/gui/graphics/sinus}
}.

\begin{minted}[mathescape,
    linenos,
    numbersep=5pt,
    gobble=2,
    fontsize=\small,
    frame=lines,
    framesep=2mm]{java}
    private void drawSinusCurve() {
        Pane sinusCanvas = sinusView.getSinusCanvas();
        double zoom = sinusView.getZoomSlider().valueProperty().get();
        double phase = sinusView.getPhaseSlider().valueProperty().get();
        double frequency = sinusView.getFrequencySlider().valueProperty().get();
        double amplitude = sinusView.getAmplitudeSlider().valueProperty().get();

        double width = sinusCanvas.widthProperty().get();
        double height = sinusCanvas.heightProperty().get();

        double start = - (width / 2);

        ObservableList<Double> points = sinusView.getSinusLine().getPoints();
        List<Double> tmpPoints = new ArrayList<>();

        for (double i = 0; i < width; i+= 1/zoom) {

            double sinX = (start+i);
            double sinY = sinusModel.getY(sinX, amplitude, frequency, phase);

            double x = (sinX*zoom - start);
            double y = (sinY*zoom + (height / 2)) ;

            tmpPoints.add(x);
            tmpPoints.add(y);
        }

        points.clear();
        points.addAll(tmpPoints);
    }
\end{minted}\\


\subsection{Anmerkung und Ergänzungen}

Die Sinuskurve wird mittels einer \code{Polyline} gezeichnet, deren Koordinaten über eine Liste von Punkten (Typ \code{double}) verwaltet werden\footnote{
s. ``Class Polyline``: \url{https://docs.oracle.com/javase/8/javafx/api/javafx/scene/shape/Polyline.html} - abgerufen 9.2.2024.
}.\\

\noindent
Die Methode ist als Observer für verschiedene Property-Änderungen registriert, also insb. die der Slider.\\

\noindent
Der Scaling-Faktor (\code{zoom}) beeinflusst die Auflösung der berechneten Werte - je höher der Scaling-Faktor, desto höher die Auflösung: Es werden dann auch mehr Werte berechnet.\\

\noindent
Für die \code{Pane}, die zum Zeichnen des Koordinatensystems und der Sinuskurve verwendet wird, sollte eine Clipping-Maske erstellt werden, damit bei einer Änderung der Amplitude / des Scaling-Faktors nicht über die \code{Pane} hinausgezeichnet wird (s. a. \cite[222]{Oec22}).\\

\noindent
Damit Kindelemente im \code{HBox}- / \code{VBox}-Layout den zur Verfügung stehenden Platz einnehmen (und dadurch ggf. automatisch vergrößert bzw. verkleinert werden), sollte  die statische Methode \code{setHgrow} (\code{HBox}) bzw. \code{setVgrow} (\code{VBox}) verwendet werden.\\
Der folgende Code führt dazu, dass der \code{Slider} als Kindelement eines \code{HBox}-Layout-Panes mit dem Elternelement mitwächst, je nach zur Verfügung stehendem Platz\footnote{
s.a. ``setHgrow``: \url{https://docs.oracle.com/javase/8/javafx/api/javafx/scene/layout/HBox.html#setHgrow-javafx.scene.Node-javafx.scene.layout.Priority-} - abgerufen 9.2.2024
}:

\begin{minted}[mathescape,
    numbersep=5pt,
    gobble=2,
    frame=none,
    framesep=2mm]{java}
    HBox.setHgrow(getAmplitudeSlider(), Priority.ALWAYS);
\end{minted}\\

\noindent
Methoden wie \code{setLabel} sucht man i.d.R. bei den meisten Controls vergeblich - Beschriftungen (in der Form von \textit{fieldLabel}s) sollten dann über extra \code{Label}s realisiert werden.\\

\noindent
Die Methoden \code{setMajorTickUnit()} und \code{setMinorTickCount()} beeinflussen das Rendering der Ticks eines Sliders\footnote{
s.a. ``Class Slider``: \url{https://docs.oracle.com/javase/8/javafx/api/javafx/scene/control/Slider.html} - abgerufen 9.2.2024
}:

\noindent
\begin{minted}[mathescape,
    numbersep=5pt,
    gobble=2,
    frame=none,
    framesep=2mm]{java}
    // Alle 2 Einheiten wird ein ``major tick`` gerendert:
    slider.setMajorTickUnit(2f);
    // Zwischen zwei major ticks sollen immer 5 ``minor ticks`` gerendert werden:
    slider.setMinorTickCount(5);
\end{minted}