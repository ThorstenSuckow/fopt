\section{Last Minute Remarks}

\begin{itemize}
    \item wenn zur Laufzeit eine Exception geworfen werden soll, und es keine weitere Aussage zu der Art der Exception gibt, eine RuntimeException nutzen.
    Ansonsten muss eine checked Exception auch in der throws Klausel der Methode deklariert werden
    \item \code{List<T>: get(index: int)}, \textit{nicht} \code{getAt}!
    \item Wenn Queues gefordert sind, bspw. bei fairen Warteschlange, das Hinzufügen und Entfernen der Objekte durchführen.
    Werden Ausnahmen geworfen, und ein Thread, der die Ausnahme geworfen hat, soll aus der Warteschlange raus, mit try-finally arbeiten!
    \item \code{BufferedReader.readLine()} liest bis zum Ende der Zeile (\textbf{\textbackslash{n}}) \textbf{oder} bis zum Ende des Input-Streams.
    Wird also nach einer Antwort verlangt, wie die Daten vom Server gelesen werden, ist es wichtig, die Implementierung des Servers zu kennen.\\
    Sendet der client über write(), und sendet der Client keine newline, gilt. o.a. Aussage.\\
    Alles weitere sollte mit dem datenstromorientierten Verhalten von TCP bzw. dem nachrichtenorientierten Charakter von UDP erklärt werden.
    \item Wenn ein Thread interruptable ist, also man einen Thread bspw. aus wait() herausholen können muss, dann ist ein try-catch um die while-schleife zu setzen, \textbf{oder} man gibt in der throws-Klausel der Methode an, dass die Exception geworfen wird und läßt den catch-Block weg.\\
    In jedem Fall muss als Schleifenbedingung \code{!isInterrupted()} mit aufgenommen werden (Konjunktion mit der regulären Wartebedingung.)
    \item Auf die Aufgabenstellung bei \code{synchronized} achten - insb, wie oft und von wem der Objektzustand geändert werden kann.
    \item Bei \textbf{MVP} laut \cite[227 f. Listing 4.8]{Oec22} meldet die View beim Presenter die Events an: Die View kennt die Methoden des Presenters und kann diese den UI-Elementen als Observer zuweisen.\\
    Die View ist auch in der Lage, den Presenter anzuweisen.\\
    Die \code{start}-Methode erzeugt das View-Objekt, ruft davon eine \code{getUI()}-Methode auf, die das \textit{root}-Element für die Scene zurückliefert, und zeigt dann die Stage an.
    \item Kurzform für Lambda-Methodenreferenz für \code{EventHandler<ActionEvent>} (Parameter für \code{setOnAction()}:\\
    \begin{minted}[]{java}
    EventHandler<ActionEvent> h = e -> presenter.doSomething();
    \end{minted}
\end{itemize}