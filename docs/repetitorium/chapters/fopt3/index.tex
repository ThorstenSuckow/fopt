
\chapter{FOPT3 - Grundlagen der Programmierung grafischer Benutzeroberflächen in Java (Grafische Benutzeroberflächen 1)}

\section{Lehrstoff}

Das Skript FOPT3 bezieht sich auf folgende Inhalte im Buch:

\begin{tcolorbox}[colback=white!20,color=white]
    \begin{enumerate}
        \setcounter{enumi}{1}
        \item \textbf{Einführung in die Programmierung grafischer Benutzeroberflächen mit Java FX}
        \begin{itemize}
            \item[] Kapitel 4
            \item[] Abschnitt 4.1
        \end{itemize}

        \item \textbf{Properties, Bindings und JavaFX-Collections}
        \begin{itemize}
            \item[] Abschnitt 4.2
        \end{itemize}

        \item \textbf{Elemente von JavaFX}
        \begin{itemize}
            \item[] Abschnitt 4.3
            \item[] Abschnitt 4.3.1
            \item[] Abschnitt 4.3.2
            \item[] Abschnitt 4.3.3
            \item[] Abschnitt 4.3.4
        \end{itemize}

    \end{enumerate}
\end{tcolorbox}

\newpage


\input{chapters/fopt3/1. Einführung in UML}


\input{chapters/fopt3/2. Einführung in JavaFX}


\newpage
\section*{Notizen}

\newpage
